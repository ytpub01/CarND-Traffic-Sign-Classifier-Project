
% Default to the notebook output style

    


% Inherit from the specified cell style.




    
\documentclass[11pt]{article}

    
    
    \usepackage[T1]{fontenc}
    % Nicer default font (+ math font) than Computer Modern for most use cases
    \usepackage{mathpazo}

    % Basic figure setup, for now with no caption control since it's done
    % automatically by Pandoc (which extracts ![](path) syntax from Markdown).
    \usepackage{graphicx}
    % We will generate all images so they have a width \maxwidth. This means
    % that they will get their normal width if they fit onto the page, but
    % are scaled down if they would overflow the margins.
    \makeatletter
    \def\maxwidth{\ifdim\Gin@nat@width>\linewidth\linewidth
    \else\Gin@nat@width\fi}
    \makeatother
    \let\Oldincludegraphics\includegraphics
    % Set max figure width to be 80% of text width, for now hardcoded.
    \renewcommand{\includegraphics}[1]{\Oldincludegraphics[width=.8\maxwidth]{#1}}
    % Ensure that by default, figures have no caption (until we provide a
    % proper Figure object with a Caption API and a way to capture that
    % in the conversion process - todo).
    \usepackage{caption}
    \DeclareCaptionLabelFormat{nolabel}{}
    \captionsetup{labelformat=nolabel}

    \usepackage{adjustbox} % Used to constrain images to a maximum size 
    \usepackage{xcolor} % Allow colors to be defined
    \usepackage{enumerate} % Needed for markdown enumerations to work
    \usepackage{geometry} % Used to adjust the document margins
    \usepackage{amsmath} % Equations
    \usepackage{amssymb} % Equations
    \usepackage{textcomp} % defines textquotesingle
    % Hack from http://tex.stackexchange.com/a/47451/13684:
    \AtBeginDocument{%
        \def\PYZsq{\textquotesingle}% Upright quotes in Pygmentized code
    }
    \usepackage{upquote} % Upright quotes for verbatim code
    \usepackage{eurosym} % defines \euro
    \usepackage[mathletters]{ucs} % Extended unicode (utf-8) support
    \usepackage[utf8x]{inputenc} % Allow utf-8 characters in the tex document
    \usepackage{fancyvrb} % verbatim replacement that allows latex
    \usepackage{grffile} % extends the file name processing of package graphics 
                         % to support a larger range 
    % The hyperref package gives us a pdf with properly built
    % internal navigation ('pdf bookmarks' for the table of contents,
    % internal cross-reference links, web links for URLs, etc.)
    \usepackage{hyperref}
    \usepackage{longtable} % longtable support required by pandoc >1.10
    \usepackage{booktabs}  % table support for pandoc > 1.12.2
    \usepackage[inline]{enumitem} % IRkernel/repr support (it uses the enumerate* environment)
    \usepackage[normalem]{ulem} % ulem is needed to support strikethroughs (\sout)
                                % normalem makes italics be italics, not underlines
    

    
    
    % Colors for the hyperref package
    \definecolor{urlcolor}{rgb}{0,.145,.698}
    \definecolor{linkcolor}{rgb}{.71,0.21,0.01}
    \definecolor{citecolor}{rgb}{.12,.54,.11}

    % ANSI colors
    \definecolor{ansi-black}{HTML}{3E424D}
    \definecolor{ansi-black-intense}{HTML}{282C36}
    \definecolor{ansi-red}{HTML}{E75C58}
    \definecolor{ansi-red-intense}{HTML}{B22B31}
    \definecolor{ansi-green}{HTML}{00A250}
    \definecolor{ansi-green-intense}{HTML}{007427}
    \definecolor{ansi-yellow}{HTML}{DDB62B}
    \definecolor{ansi-yellow-intense}{HTML}{B27D12}
    \definecolor{ansi-blue}{HTML}{208FFB}
    \definecolor{ansi-blue-intense}{HTML}{0065CA}
    \definecolor{ansi-magenta}{HTML}{D160C4}
    \definecolor{ansi-magenta-intense}{HTML}{A03196}
    \definecolor{ansi-cyan}{HTML}{60C6C8}
    \definecolor{ansi-cyan-intense}{HTML}{258F8F}
    \definecolor{ansi-white}{HTML}{C5C1B4}
    \definecolor{ansi-white-intense}{HTML}{A1A6B2}

    % commands and environments needed by pandoc snippets
    % extracted from the output of `pandoc -s`
    \providecommand{\tightlist}{%
      \setlength{\itemsep}{0pt}\setlength{\parskip}{0pt}}
    \DefineVerbatimEnvironment{Highlighting}{Verbatim}{commandchars=\\\{\}}
    % Add ',fontsize=\small' for more characters per line
    \newenvironment{Shaded}{}{}
    \newcommand{\KeywordTok}[1]{\textcolor[rgb]{0.00,0.44,0.13}{\textbf{{#1}}}}
    \newcommand{\DataTypeTok}[1]{\textcolor[rgb]{0.56,0.13,0.00}{{#1}}}
    \newcommand{\DecValTok}[1]{\textcolor[rgb]{0.25,0.63,0.44}{{#1}}}
    \newcommand{\BaseNTok}[1]{\textcolor[rgb]{0.25,0.63,0.44}{{#1}}}
    \newcommand{\FloatTok}[1]{\textcolor[rgb]{0.25,0.63,0.44}{{#1}}}
    \newcommand{\CharTok}[1]{\textcolor[rgb]{0.25,0.44,0.63}{{#1}}}
    \newcommand{\StringTok}[1]{\textcolor[rgb]{0.25,0.44,0.63}{{#1}}}
    \newcommand{\CommentTok}[1]{\textcolor[rgb]{0.38,0.63,0.69}{\textit{{#1}}}}
    \newcommand{\OtherTok}[1]{\textcolor[rgb]{0.00,0.44,0.13}{{#1}}}
    \newcommand{\AlertTok}[1]{\textcolor[rgb]{1.00,0.00,0.00}{\textbf{{#1}}}}
    \newcommand{\FunctionTok}[1]{\textcolor[rgb]{0.02,0.16,0.49}{{#1}}}
    \newcommand{\RegionMarkerTok}[1]{{#1}}
    \newcommand{\ErrorTok}[1]{\textcolor[rgb]{1.00,0.00,0.00}{\textbf{{#1}}}}
    \newcommand{\NormalTok}[1]{{#1}}
    
    % Additional commands for more recent versions of Pandoc
    \newcommand{\ConstantTok}[1]{\textcolor[rgb]{0.53,0.00,0.00}{{#1}}}
    \newcommand{\SpecialCharTok}[1]{\textcolor[rgb]{0.25,0.44,0.63}{{#1}}}
    \newcommand{\VerbatimStringTok}[1]{\textcolor[rgb]{0.25,0.44,0.63}{{#1}}}
    \newcommand{\SpecialStringTok}[1]{\textcolor[rgb]{0.73,0.40,0.53}{{#1}}}
    \newcommand{\ImportTok}[1]{{#1}}
    \newcommand{\DocumentationTok}[1]{\textcolor[rgb]{0.73,0.13,0.13}{\textit{{#1}}}}
    \newcommand{\AnnotationTok}[1]{\textcolor[rgb]{0.38,0.63,0.69}{\textbf{\textit{{#1}}}}}
    \newcommand{\CommentVarTok}[1]{\textcolor[rgb]{0.38,0.63,0.69}{\textbf{\textit{{#1}}}}}
    \newcommand{\VariableTok}[1]{\textcolor[rgb]{0.10,0.09,0.49}{{#1}}}
    \newcommand{\ControlFlowTok}[1]{\textcolor[rgb]{0.00,0.44,0.13}{\textbf{{#1}}}}
    \newcommand{\OperatorTok}[1]{\textcolor[rgb]{0.40,0.40,0.40}{{#1}}}
    \newcommand{\BuiltInTok}[1]{{#1}}
    \newcommand{\ExtensionTok}[1]{{#1}}
    \newcommand{\PreprocessorTok}[1]{\textcolor[rgb]{0.74,0.48,0.00}{{#1}}}
    \newcommand{\AttributeTok}[1]{\textcolor[rgb]{0.49,0.56,0.16}{{#1}}}
    \newcommand{\InformationTok}[1]{\textcolor[rgb]{0.38,0.63,0.69}{\textbf{\textit{{#1}}}}}
    \newcommand{\WarningTok}[1]{\textcolor[rgb]{0.38,0.63,0.69}{\textbf{\textit{{#1}}}}}
    
    
    % Define a nice break command that doesn't care if a line doesn't already
    % exist.
    \def\br{\hspace*{\fill} \\* }
    % Math Jax compatability definitions
    \def\gt{>}
    \def\lt{<}
    % Document parameters
    \title{Traffic\_Sign\_Classifier}
    
    
    

    % Pygments definitions
    
\makeatletter
\def\PY@reset{\let\PY@it=\relax \let\PY@bf=\relax%
    \let\PY@ul=\relax \let\PY@tc=\relax%
    \let\PY@bc=\relax \let\PY@ff=\relax}
\def\PY@tok#1{\csname PY@tok@#1\endcsname}
\def\PY@toks#1+{\ifx\relax#1\empty\else%
    \PY@tok{#1}\expandafter\PY@toks\fi}
\def\PY@do#1{\PY@bc{\PY@tc{\PY@ul{%
    \PY@it{\PY@bf{\PY@ff{#1}}}}}}}
\def\PY#1#2{\PY@reset\PY@toks#1+\relax+\PY@do{#2}}

\expandafter\def\csname PY@tok@sc\endcsname{\def\PY@tc##1{\textcolor[rgb]{0.73,0.13,0.13}{##1}}}
\expandafter\def\csname PY@tok@s2\endcsname{\def\PY@tc##1{\textcolor[rgb]{0.73,0.13,0.13}{##1}}}
\expandafter\def\csname PY@tok@cm\endcsname{\let\PY@it=\textit\def\PY@tc##1{\textcolor[rgb]{0.25,0.50,0.50}{##1}}}
\expandafter\def\csname PY@tok@s1\endcsname{\def\PY@tc##1{\textcolor[rgb]{0.73,0.13,0.13}{##1}}}
\expandafter\def\csname PY@tok@mh\endcsname{\def\PY@tc##1{\textcolor[rgb]{0.40,0.40,0.40}{##1}}}
\expandafter\def\csname PY@tok@mb\endcsname{\def\PY@tc##1{\textcolor[rgb]{0.40,0.40,0.40}{##1}}}
\expandafter\def\csname PY@tok@si\endcsname{\let\PY@bf=\textbf\def\PY@tc##1{\textcolor[rgb]{0.73,0.40,0.53}{##1}}}
\expandafter\def\csname PY@tok@cp\endcsname{\def\PY@tc##1{\textcolor[rgb]{0.74,0.48,0.00}{##1}}}
\expandafter\def\csname PY@tok@ss\endcsname{\def\PY@tc##1{\textcolor[rgb]{0.10,0.09,0.49}{##1}}}
\expandafter\def\csname PY@tok@w\endcsname{\def\PY@tc##1{\textcolor[rgb]{0.73,0.73,0.73}{##1}}}
\expandafter\def\csname PY@tok@err\endcsname{\def\PY@bc##1{\setlength{\fboxsep}{0pt}\fcolorbox[rgb]{1.00,0.00,0.00}{1,1,1}{\strut ##1}}}
\expandafter\def\csname PY@tok@kp\endcsname{\def\PY@tc##1{\textcolor[rgb]{0.00,0.50,0.00}{##1}}}
\expandafter\def\csname PY@tok@gi\endcsname{\def\PY@tc##1{\textcolor[rgb]{0.00,0.63,0.00}{##1}}}
\expandafter\def\csname PY@tok@gt\endcsname{\def\PY@tc##1{\textcolor[rgb]{0.00,0.27,0.87}{##1}}}
\expandafter\def\csname PY@tok@ni\endcsname{\let\PY@bf=\textbf\def\PY@tc##1{\textcolor[rgb]{0.60,0.60,0.60}{##1}}}
\expandafter\def\csname PY@tok@gh\endcsname{\let\PY@bf=\textbf\def\PY@tc##1{\textcolor[rgb]{0.00,0.00,0.50}{##1}}}
\expandafter\def\csname PY@tok@ch\endcsname{\let\PY@it=\textit\def\PY@tc##1{\textcolor[rgb]{0.25,0.50,0.50}{##1}}}
\expandafter\def\csname PY@tok@na\endcsname{\def\PY@tc##1{\textcolor[rgb]{0.49,0.56,0.16}{##1}}}
\expandafter\def\csname PY@tok@nl\endcsname{\def\PY@tc##1{\textcolor[rgb]{0.63,0.63,0.00}{##1}}}
\expandafter\def\csname PY@tok@sr\endcsname{\def\PY@tc##1{\textcolor[rgb]{0.73,0.40,0.53}{##1}}}
\expandafter\def\csname PY@tok@mo\endcsname{\def\PY@tc##1{\textcolor[rgb]{0.40,0.40,0.40}{##1}}}
\expandafter\def\csname PY@tok@mi\endcsname{\def\PY@tc##1{\textcolor[rgb]{0.40,0.40,0.40}{##1}}}
\expandafter\def\csname PY@tok@vi\endcsname{\def\PY@tc##1{\textcolor[rgb]{0.10,0.09,0.49}{##1}}}
\expandafter\def\csname PY@tok@no\endcsname{\def\PY@tc##1{\textcolor[rgb]{0.53,0.00,0.00}{##1}}}
\expandafter\def\csname PY@tok@dl\endcsname{\def\PY@tc##1{\textcolor[rgb]{0.73,0.13,0.13}{##1}}}
\expandafter\def\csname PY@tok@ow\endcsname{\let\PY@bf=\textbf\def\PY@tc##1{\textcolor[rgb]{0.67,0.13,1.00}{##1}}}
\expandafter\def\csname PY@tok@kt\endcsname{\def\PY@tc##1{\textcolor[rgb]{0.69,0.00,0.25}{##1}}}
\expandafter\def\csname PY@tok@vc\endcsname{\def\PY@tc##1{\textcolor[rgb]{0.10,0.09,0.49}{##1}}}
\expandafter\def\csname PY@tok@fm\endcsname{\def\PY@tc##1{\textcolor[rgb]{0.00,0.00,1.00}{##1}}}
\expandafter\def\csname PY@tok@ge\endcsname{\let\PY@it=\textit}
\expandafter\def\csname PY@tok@gp\endcsname{\let\PY@bf=\textbf\def\PY@tc##1{\textcolor[rgb]{0.00,0.00,0.50}{##1}}}
\expandafter\def\csname PY@tok@gs\endcsname{\let\PY@bf=\textbf}
\expandafter\def\csname PY@tok@nf\endcsname{\def\PY@tc##1{\textcolor[rgb]{0.00,0.00,1.00}{##1}}}
\expandafter\def\csname PY@tok@gu\endcsname{\let\PY@bf=\textbf\def\PY@tc##1{\textcolor[rgb]{0.50,0.00,0.50}{##1}}}
\expandafter\def\csname PY@tok@kr\endcsname{\let\PY@bf=\textbf\def\PY@tc##1{\textcolor[rgb]{0.00,0.50,0.00}{##1}}}
\expandafter\def\csname PY@tok@ne\endcsname{\let\PY@bf=\textbf\def\PY@tc##1{\textcolor[rgb]{0.82,0.25,0.23}{##1}}}
\expandafter\def\csname PY@tok@nn\endcsname{\let\PY@bf=\textbf\def\PY@tc##1{\textcolor[rgb]{0.00,0.00,1.00}{##1}}}
\expandafter\def\csname PY@tok@gr\endcsname{\def\PY@tc##1{\textcolor[rgb]{1.00,0.00,0.00}{##1}}}
\expandafter\def\csname PY@tok@nt\endcsname{\let\PY@bf=\textbf\def\PY@tc##1{\textcolor[rgb]{0.00,0.50,0.00}{##1}}}
\expandafter\def\csname PY@tok@vm\endcsname{\def\PY@tc##1{\textcolor[rgb]{0.10,0.09,0.49}{##1}}}
\expandafter\def\csname PY@tok@nc\endcsname{\let\PY@bf=\textbf\def\PY@tc##1{\textcolor[rgb]{0.00,0.00,1.00}{##1}}}
\expandafter\def\csname PY@tok@bp\endcsname{\def\PY@tc##1{\textcolor[rgb]{0.00,0.50,0.00}{##1}}}
\expandafter\def\csname PY@tok@cs\endcsname{\let\PY@it=\textit\def\PY@tc##1{\textcolor[rgb]{0.25,0.50,0.50}{##1}}}
\expandafter\def\csname PY@tok@cpf\endcsname{\let\PY@it=\textit\def\PY@tc##1{\textcolor[rgb]{0.25,0.50,0.50}{##1}}}
\expandafter\def\csname PY@tok@nv\endcsname{\def\PY@tc##1{\textcolor[rgb]{0.10,0.09,0.49}{##1}}}
\expandafter\def\csname PY@tok@m\endcsname{\def\PY@tc##1{\textcolor[rgb]{0.40,0.40,0.40}{##1}}}
\expandafter\def\csname PY@tok@c1\endcsname{\let\PY@it=\textit\def\PY@tc##1{\textcolor[rgb]{0.25,0.50,0.50}{##1}}}
\expandafter\def\csname PY@tok@gd\endcsname{\def\PY@tc##1{\textcolor[rgb]{0.63,0.00,0.00}{##1}}}
\expandafter\def\csname PY@tok@go\endcsname{\def\PY@tc##1{\textcolor[rgb]{0.53,0.53,0.53}{##1}}}
\expandafter\def\csname PY@tok@kn\endcsname{\let\PY@bf=\textbf\def\PY@tc##1{\textcolor[rgb]{0.00,0.50,0.00}{##1}}}
\expandafter\def\csname PY@tok@sh\endcsname{\def\PY@tc##1{\textcolor[rgb]{0.73,0.13,0.13}{##1}}}
\expandafter\def\csname PY@tok@o\endcsname{\def\PY@tc##1{\textcolor[rgb]{0.40,0.40,0.40}{##1}}}
\expandafter\def\csname PY@tok@s\endcsname{\def\PY@tc##1{\textcolor[rgb]{0.73,0.13,0.13}{##1}}}
\expandafter\def\csname PY@tok@vg\endcsname{\def\PY@tc##1{\textcolor[rgb]{0.10,0.09,0.49}{##1}}}
\expandafter\def\csname PY@tok@sd\endcsname{\let\PY@it=\textit\def\PY@tc##1{\textcolor[rgb]{0.73,0.13,0.13}{##1}}}
\expandafter\def\csname PY@tok@k\endcsname{\let\PY@bf=\textbf\def\PY@tc##1{\textcolor[rgb]{0.00,0.50,0.00}{##1}}}
\expandafter\def\csname PY@tok@kd\endcsname{\let\PY@bf=\textbf\def\PY@tc##1{\textcolor[rgb]{0.00,0.50,0.00}{##1}}}
\expandafter\def\csname PY@tok@sx\endcsname{\def\PY@tc##1{\textcolor[rgb]{0.00,0.50,0.00}{##1}}}
\expandafter\def\csname PY@tok@se\endcsname{\let\PY@bf=\textbf\def\PY@tc##1{\textcolor[rgb]{0.73,0.40,0.13}{##1}}}
\expandafter\def\csname PY@tok@nd\endcsname{\def\PY@tc##1{\textcolor[rgb]{0.67,0.13,1.00}{##1}}}
\expandafter\def\csname PY@tok@nb\endcsname{\def\PY@tc##1{\textcolor[rgb]{0.00,0.50,0.00}{##1}}}
\expandafter\def\csname PY@tok@c\endcsname{\let\PY@it=\textit\def\PY@tc##1{\textcolor[rgb]{0.25,0.50,0.50}{##1}}}
\expandafter\def\csname PY@tok@sb\endcsname{\def\PY@tc##1{\textcolor[rgb]{0.73,0.13,0.13}{##1}}}
\expandafter\def\csname PY@tok@mf\endcsname{\def\PY@tc##1{\textcolor[rgb]{0.40,0.40,0.40}{##1}}}
\expandafter\def\csname PY@tok@kc\endcsname{\let\PY@bf=\textbf\def\PY@tc##1{\textcolor[rgb]{0.00,0.50,0.00}{##1}}}
\expandafter\def\csname PY@tok@il\endcsname{\def\PY@tc##1{\textcolor[rgb]{0.40,0.40,0.40}{##1}}}
\expandafter\def\csname PY@tok@sa\endcsname{\def\PY@tc##1{\textcolor[rgb]{0.73,0.13,0.13}{##1}}}

\def\PYZbs{\char`\\}
\def\PYZus{\char`\_}
\def\PYZob{\char`\{}
\def\PYZcb{\char`\}}
\def\PYZca{\char`\^}
\def\PYZam{\char`\&}
\def\PYZlt{\char`\<}
\def\PYZgt{\char`\>}
\def\PYZsh{\char`\#}
\def\PYZpc{\char`\%}
\def\PYZdl{\char`\$}
\def\PYZhy{\char`\-}
\def\PYZsq{\char`\'}
\def\PYZdq{\char`\"}
\def\PYZti{\char`\~}
% for compatibility with earlier versions
\def\PYZat{@}
\def\PYZlb{[}
\def\PYZrb{]}
\makeatother


    % Exact colors from NB
    \definecolor{incolor}{rgb}{0.0, 0.0, 0.5}
    \definecolor{outcolor}{rgb}{0.545, 0.0, 0.0}



    
    % Prevent overflowing lines due to hard-to-break entities
    \sloppy 
    % Setup hyperref package
    \hypersetup{
      breaklinks=true,  % so long urls are correctly broken across lines
      colorlinks=true,
      urlcolor=urlcolor,
      linkcolor=linkcolor,
      citecolor=citecolor,
      }
    % Slightly bigger margins than the latex defaults
    
    \geometry{verbose,tmargin=1in,bmargin=1in,lmargin=1in,rmargin=1in}
    
    

    \begin{document}
    
    
    \maketitle
    
    

    
    \section{Self-Driving Car Engineer
Nanodegree}\label{self-driving-car-engineer-nanodegree}

\subsection{Deep Learning}\label{deep-learning}

\subsection{Project: Build a Traffic Sign Recognition
Classifier}\label{project-build-a-traffic-sign-recognition-classifier}

In this notebook, a template is provided for you to implement your
functionality in stages, which is required to successfully complete this
project. If additional code is required that cannot be included in the
notebook, be sure that the Python code is successfully imported and
included in your submission if necessary.

\begin{quote}
\textbf{Note}: Once you have completed all of the code implementations,
you need to finalize your work by exporting the iPython Notebook as an
HTML document. Before exporting the notebook to html, all of the code
cells need to have been run so that reviewers can see the final
implementation and output. You can then export the notebook by using the
menu above and navigating to \n", "\textbf{File -\textgreater{} Download
as -\textgreater{} HTML (.html)}. Include the finished document along
with this notebook as your submission.
\end{quote}

In addition to implementing code, there is a writeup to complete. The
writeup should be completed in a separate file, which can be either a
markdown file or a pdf document. There is a
\href{https://github.com/udacity/CarND-Traffic-Sign-Classifier-Project/blob/master/writeup_template.md}{write
up template} that can be used to guide the writing process. Completing
the code template and writeup template will cover all of the
\href{https://review.udacity.com/\#!/rubrics/481/view}{rubric points}
for this project.

The \href{https://review.udacity.com/\#!/rubrics/481/view}{rubric}
contains "Stand Out Suggestions" for enhancing the project beyond the
minimum requirements. The stand out suggestions are optional. If you
decide to pursue the "stand out suggestions", you can include the code
in this Ipython notebook and also discuss the results in the writeup
file.

\begin{quote}
\textbf{Note:} Code and Markdown cells can be executed using the
\textbf{Shift + Enter} keyboard shortcut. In addition, Markdown cells
can be edited by typically double-clicking the cell to enter edit mode.
\end{quote}

    \begin{Verbatim}[commandchars=\\\{\}]
{\color{incolor}In [{\color{incolor}1}]:} \PY{c+c1}{\PYZsh{} Load pickled data}
        \PY{k+kn}{import} \PY{n+nn}{pickle}
        
        \PY{c+c1}{\PYZsh{} TODO: Fill this in based on where you saved the training and testing data}
        
        \PY{n}{training\PYZus{}file} \PY{o}{=} \PY{l+s+s1}{\PYZsq{}}\PY{l+s+s1}{dataset/train.p}\PY{l+s+s1}{\PYZsq{}}
        \PY{n}{validation\PYZus{}file} \PY{o}{=} \PY{l+s+s1}{\PYZsq{}}\PY{l+s+s1}{dataset/valid.p}\PY{l+s+s1}{\PYZsq{}}
        \PY{n}{testing\PYZus{}file} \PY{o}{=} \PY{l+s+s1}{\PYZsq{}}\PY{l+s+s1}{dataset/test.p}\PY{l+s+s1}{\PYZsq{}}
        
        \PY{k}{with} \PY{n+nb}{open}\PY{p}{(}\PY{n}{training\PYZus{}file}\PY{p}{,} \PY{n}{mode}\PY{o}{=}\PY{l+s+s1}{\PYZsq{}}\PY{l+s+s1}{rb}\PY{l+s+s1}{\PYZsq{}}\PY{p}{)} \PY{k}{as} \PY{n}{f}\PY{p}{:}
            \PY{n}{train} \PY{o}{=} \PY{n}{pickle}\PY{o}{.}\PY{n}{load}\PY{p}{(}\PY{n}{f}\PY{p}{)}
        \PY{k}{with} \PY{n+nb}{open}\PY{p}{(}\PY{n}{validation\PYZus{}file}\PY{p}{,} \PY{n}{mode}\PY{o}{=}\PY{l+s+s1}{\PYZsq{}}\PY{l+s+s1}{rb}\PY{l+s+s1}{\PYZsq{}}\PY{p}{)} \PY{k}{as} \PY{n}{f}\PY{p}{:}
            \PY{n}{valid} \PY{o}{=} \PY{n}{pickle}\PY{o}{.}\PY{n}{load}\PY{p}{(}\PY{n}{f}\PY{p}{)}
        \PY{k}{with} \PY{n+nb}{open}\PY{p}{(}\PY{n}{testing\PYZus{}file}\PY{p}{,} \PY{n}{mode}\PY{o}{=}\PY{l+s+s1}{\PYZsq{}}\PY{l+s+s1}{rb}\PY{l+s+s1}{\PYZsq{}}\PY{p}{)} \PY{k}{as} \PY{n}{f}\PY{p}{:}
            \PY{n}{test} \PY{o}{=} \PY{n}{pickle}\PY{o}{.}\PY{n}{load}\PY{p}{(}\PY{n}{f}\PY{p}{)}
            
        \PY{n}{X\PYZus{}train}\PY{p}{,} \PY{n}{y\PYZus{}train} \PY{o}{=} \PY{n}{train}\PY{p}{[}\PY{l+s+s1}{\PYZsq{}}\PY{l+s+s1}{features}\PY{l+s+s1}{\PYZsq{}}\PY{p}{]}\PY{p}{,} \PY{n}{train}\PY{p}{[}\PY{l+s+s1}{\PYZsq{}}\PY{l+s+s1}{labels}\PY{l+s+s1}{\PYZsq{}}\PY{p}{]}
        \PY{n}{X\PYZus{}valid}\PY{p}{,} \PY{n}{y\PYZus{}valid} \PY{o}{=} \PY{n}{valid}\PY{p}{[}\PY{l+s+s1}{\PYZsq{}}\PY{l+s+s1}{features}\PY{l+s+s1}{\PYZsq{}}\PY{p}{]}\PY{p}{,} \PY{n}{valid}\PY{p}{[}\PY{l+s+s1}{\PYZsq{}}\PY{l+s+s1}{labels}\PY{l+s+s1}{\PYZsq{}}\PY{p}{]}
        \PY{n}{X\PYZus{}test}\PY{p}{,} \PY{n}{y\PYZus{}test} \PY{o}{=} \PY{n}{test}\PY{p}{[}\PY{l+s+s1}{\PYZsq{}}\PY{l+s+s1}{features}\PY{l+s+s1}{\PYZsq{}}\PY{p}{]}\PY{p}{,} \PY{n}{test}\PY{p}{[}\PY{l+s+s1}{\PYZsq{}}\PY{l+s+s1}{labels}\PY{l+s+s1}{\PYZsq{}}\PY{p}{]}
        
        \PY{k}{assert}\PY{p}{(}\PY{n+nb}{len}\PY{p}{(}\PY{n}{X\PYZus{}train}\PY{p}{)} \PY{o}{==} \PY{n+nb}{len}\PY{p}{(}\PY{n}{y\PYZus{}train}\PY{p}{)}\PY{p}{)}
        \PY{k}{assert}\PY{p}{(}\PY{n+nb}{len}\PY{p}{(}\PY{n}{X\PYZus{}valid}\PY{p}{)} \PY{o}{==} \PY{n+nb}{len}\PY{p}{(}\PY{n}{y\PYZus{}valid}\PY{p}{)}\PY{p}{)}
        \PY{k}{assert}\PY{p}{(}\PY{n+nb}{len}\PY{p}{(}\PY{n}{X\PYZus{}test}\PY{p}{)} \PY{o}{==} \PY{n+nb}{len}\PY{p}{(}\PY{n}{y\PYZus{}test}\PY{p}{)}\PY{p}{)}
\end{Verbatim}

    \begin{center}\rule{0.5\linewidth}{\linethickness}\end{center}

\subsection{Step 0: Load The Data}\label{step-0-load-the-data}

    \begin{center}\rule{0.5\linewidth}{\linethickness}\end{center}

\subsection{Step 1: Dataset Summary \&
Exploration}\label{step-1-dataset-summary-exploration}

The pickled data is a dictionary with 4 key/value pairs:

\begin{itemize}
\tightlist
\item
  \texttt{\textquotesingle{}features\textquotesingle{}} is a 4D array
  containing raw pixel data of the traffic sign images, (num examples,
  width, height, channels).
\item
  \texttt{\textquotesingle{}labels\textquotesingle{}} is a 1D array
  containing the label/class id of the traffic sign. The file
  \texttt{signnames.csv} contains id -\textgreater{} name mappings for
  each id.
\item
  \texttt{\textquotesingle{}sizes\textquotesingle{}} is a list
  containing tuples, (width, height) representing the original width and
  height the image.
\item
  \texttt{\textquotesingle{}coords\textquotesingle{}} is a list
  containing tuples, (x1, y1, x2, y2) representing coordinates of a
  bounding box around the sign in the image. \textbf{THESE COORDINATES
  ASSUME THE ORIGINAL IMAGE. THE PICKLED DATA CONTAINS RESIZED VERSIONS
  (32 by 32) OF THESE IMAGES}
\end{itemize}

Complete the basic data summary below. Use python, numpy and/or pandas
methods to calculate the data summary rather than hard coding the
results. For example, the
\href{http://pandas.pydata.org/pandas-docs/stable/generated/pandas.DataFrame.shape.html}{pandas
shape method} might be useful for calculating some of the summary
results.

    \subsubsection{Provide a Basic Summary of the Data Set Using Python,
Numpy and/or
Pandas}\label{provide-a-basic-summary-of-the-data-set-using-python-numpy-andor-pandas}

    \begin{Verbatim}[commandchars=\\\{\}]
{\color{incolor}In [{\color{incolor}2}]:} \PY{c+c1}{\PYZsh{}\PYZsh{}\PYZsh{} Replace each question mark with the appropriate value. }
        \PY{c+c1}{\PYZsh{}\PYZsh{}\PYZsh{} Use python, pandas or numpy methods rather than hard coding the results}
        
        \PY{k+kn}{import} \PY{n+nn}{tensorflow} \PY{k}{as} \PY{n+nn}{tf}
        
        \PY{n}{EPOCHS} \PY{o}{=} \PY{l+m+mi}{10}
        \PY{n}{BATCH\PYZus{}SIZE} \PY{o}{=} \PY{l+m+mi}{128}
        
        \PY{c+c1}{\PYZsh{} TODO: Number of training examples}
        \PY{n}{n\PYZus{}train} \PY{o}{=} \PY{n+nb}{len}\PY{p}{(}\PY{n}{X\PYZus{}train}\PY{p}{)}
        
        \PY{c+c1}{\PYZsh{} TODO: Number of validation examples}
        \PY{n}{n\PYZus{}valid} \PY{o}{=} \PY{n+nb}{len}\PY{p}{(}\PY{n}{X\PYZus{}valid}\PY{p}{)}
        
        \PY{c+c1}{\PYZsh{} TODO: Number of testing examples.}
        \PY{n}{n\PYZus{}test} \PY{o}{=} \PY{n+nb}{len}\PY{p}{(}\PY{n}{X\PYZus{}test}\PY{p}{)}
        
        \PY{c+c1}{\PYZsh{} TODO: What\PYZsq{}s the shape of an traffic sign image?}
        \PY{n}{image\PYZus{}shape} \PY{o}{=} \PY{n}{X\PYZus{}train}\PY{p}{[}\PY{l+m+mi}{0}\PY{p}{]}\PY{o}{.}\PY{n}{shape}
        
        \PY{c+c1}{\PYZsh{} TODO: How many unique classes/labels there are in the dataset.}
        \PY{n}{n\PYZus{}classes} \PY{o}{=} \PY{n+nb}{len}\PY{p}{(}\PY{n+nb}{set}\PY{p}{(}\PY{n}{train}\PY{p}{[}\PY{l+s+s1}{\PYZsq{}}\PY{l+s+s1}{labels}\PY{l+s+s1}{\PYZsq{}}\PY{p}{]}\PY{p}{)}\PY{p}{)}
        
        \PY{n+nb}{print}\PY{p}{(}\PY{l+s+s2}{\PYZdq{}}\PY{l+s+s2}{Number of training examples =}\PY{l+s+s2}{\PYZdq{}}\PY{p}{,} \PY{n}{n\PYZus{}train}\PY{p}{)}
        \PY{n+nb}{print}\PY{p}{(}\PY{l+s+s2}{\PYZdq{}}\PY{l+s+s2}{Number of validation examples =}\PY{l+s+s2}{\PYZdq{}}\PY{p}{,} \PY{n}{n\PYZus{}valid}\PY{p}{)}
        \PY{n+nb}{print}\PY{p}{(}\PY{l+s+s2}{\PYZdq{}}\PY{l+s+s2}{Number of testing examples =}\PY{l+s+s2}{\PYZdq{}}\PY{p}{,} \PY{n}{n\PYZus{}test}\PY{p}{)}
        \PY{n+nb}{print}\PY{p}{(}\PY{l+s+s2}{\PYZdq{}}\PY{l+s+s2}{Image data shape =}\PY{l+s+s2}{\PYZdq{}}\PY{p}{,} \PY{n}{image\PYZus{}shape}\PY{p}{)}
        \PY{n+nb}{print}\PY{p}{(}\PY{l+s+s2}{\PYZdq{}}\PY{l+s+s2}{Number of classes =}\PY{l+s+s2}{\PYZdq{}}\PY{p}{,} \PY{n}{n\PYZus{}classes}\PY{p}{)}
\end{Verbatim}

    \begin{Verbatim}[commandchars=\\\{\}]
Number of training examples = 34799
Number of validation examples = 4410
Number of testing examples = 12630
Image data shape = (32, 32, 3)
Number of classes = 43

    \end{Verbatim}

    \subsubsection{Include an exploratory visualization of the
dataset}\label{include-an-exploratory-visualization-of-the-dataset}

    Visualize the German Traffic Signs Dataset using the pickled file(s).
This is open ended, suggestions include: plotting traffic sign images,
plotting the count of each sign, etc.

The \href{http://matplotlib.org/}{Matplotlib}
\href{http://matplotlib.org/examples/index.html}{examples} and
\href{http://matplotlib.org/gallery.html}{gallery} pages are a great
resource for doing visualizations in Python.

\textbf{NOTE:} It's recommended you start with something simple first.
If you wish to do more, come back to it after you've completed the rest
of the sections. It can be interesting to look at the distribution of
classes in the training, validation and test set. Is the distribution
the same? Are there more examples of some classes than others?

    \begin{Verbatim}[commandchars=\\\{\}]
{\color{incolor}In [{\color{incolor}3}]:} \PY{k+kn}{import} \PY{n+nn}{numpy} \PY{k}{as} \PY{n+nn}{np}
        \PY{k+kn}{import} \PY{n+nn}{matplotlib}\PY{n+nn}{.}\PY{n+nn}{pyplot} \PY{k}{as} \PY{n+nn}{plt}
        
        
        \PY{n}{fig} \PY{o}{=} \PY{n}{plt}\PY{o}{.}\PY{n}{figure}\PY{p}{(}\PY{p}{)}
        \PY{n}{atrain} \PY{o}{=} \PY{n}{fig}\PY{o}{.}\PY{n}{add\PYZus{}subplot}\PY{p}{(}\PY{l+m+mi}{311}\PY{p}{)}
        \PY{n}{atest} \PY{o}{=} \PY{n}{fig}\PY{o}{.}\PY{n}{add\PYZus{}subplot}\PY{p}{(}\PY{l+m+mi}{312}\PY{p}{)}
        \PY{n}{avalid} \PY{o}{=} \PY{n}{fig}\PY{o}{.}\PY{n}{add\PYZus{}subplot}\PY{p}{(}\PY{l+m+mi}{313}\PY{p}{)}
        
        \PY{n}{numBins} \PY{o}{=} \PY{l+m+mi}{20}
        \PY{n}{atrain}\PY{o}{.}\PY{n}{hist}\PY{p}{(}\PY{n}{y\PYZus{}train}\PY{p}{,}\PY{n}{numBins}\PY{p}{,}\PY{n}{color}\PY{o}{=}\PY{l+s+s1}{\PYZsq{}}\PY{l+s+s1}{green}\PY{l+s+s1}{\PYZsq{}}\PY{p}{,}\PY{n}{alpha}\PY{o}{=}\PY{l+m+mf}{0.8}\PY{p}{)}
        \PY{n}{atrain}\PY{o}{.}\PY{n}{set\PYZus{}title}\PY{p}{(}\PY{l+s+s2}{\PYZdq{}}\PY{l+s+s2}{Training set}\PY{l+s+s2}{\PYZdq{}}\PY{p}{)}
        \PY{n}{atest}\PY{o}{.}\PY{n}{hist}\PY{p}{(}\PY{n}{y\PYZus{}test}\PY{p}{,}\PY{n}{numBins}\PY{p}{,}\PY{n}{color}\PY{o}{=}\PY{l+s+s1}{\PYZsq{}}\PY{l+s+s1}{blue}\PY{l+s+s1}{\PYZsq{}}\PY{p}{,}\PY{n}{alpha}\PY{o}{=}\PY{l+m+mf}{0.8}\PY{p}{)}
        \PY{n}{atest}\PY{o}{.}\PY{n}{set\PYZus{}title}\PY{p}{(}\PY{l+s+s2}{\PYZdq{}}\PY{l+s+s2}{Test set}\PY{l+s+s2}{\PYZdq{}}\PY{p}{)}
        \PY{n}{avalid}\PY{o}{.}\PY{n}{hist}\PY{p}{(}\PY{n}{y\PYZus{}valid}\PY{p}{,}\PY{n}{numBins}\PY{p}{,}\PY{n}{color}\PY{o}{=}\PY{l+s+s1}{\PYZsq{}}\PY{l+s+s1}{black}\PY{l+s+s1}{\PYZsq{}}\PY{p}{,}\PY{n}{alpha}\PY{o}{=}\PY{l+m+mf}{0.8}\PY{p}{)}
        \PY{n}{avalid}\PY{o}{.}\PY{n}{set\PYZus{}title}\PY{p}{(}\PY{l+s+s2}{\PYZdq{}}\PY{l+s+s2}{Validation set}\PY{l+s+s2}{\PYZdq{}}\PY{p}{)}
        \PY{n}{plt}\PY{o}{.}\PY{n}{show}\PY{p}{(}\PY{p}{)}
\end{Verbatim}

    \begin{center}
    \adjustimage{max size={0.9\linewidth}{0.9\paperheight}}{Traffic_Sign_Classifier_files/Traffic_Sign_Classifier_8_0.png}
    \end{center}
    { \hspace*{\fill} \\}
    
    \begin{Verbatim}[commandchars=\\\{\}]
{\color{incolor}In [{\color{incolor}4}]:} \PY{c+c1}{\PYZsh{}\PYZsh{}\PYZsh{} Data exploration visualization code goes here.}
        \PY{c+c1}{\PYZsh{}\PYZsh{}\PYZsh{} Feel free to use as many code cells as needed.}
        
        \PY{k+kn}{import} \PY{n+nn}{random}
        \PY{n}{index} \PY{o}{=} \PY{n}{random}\PY{o}{.}\PY{n}{randint}\PY{p}{(}\PY{l+m+mi}{0}\PY{p}{,} \PY{n+nb}{len}\PY{p}{(}\PY{n}{X\PYZus{}train}\PY{p}{)}\PY{o}{\PYZhy{}}\PY{l+m+mi}{1}\PY{p}{)}
        \PY{n}{image} \PY{o}{=} \PY{n}{X\PYZus{}train}\PY{p}{[}\PY{n}{index}\PY{p}{]}
        
        \PY{k+kn}{import} \PY{n+nn}{matplotlib}\PY{n+nn}{.}\PY{n+nn}{pyplot} \PY{k}{as} \PY{n+nn}{plt}
        \PY{c+c1}{\PYZsh{} Visualizations will be shown in the notebook.}
        \PY{o}{\PYZpc{}}\PY{k}{matplotlib} inline
        
        \PY{n}{plt}\PY{o}{.}\PY{n}{figure}\PY{p}{(}\PY{n}{figsize}\PY{o}{=}\PY{p}{(}\PY{l+m+mi}{1}\PY{p}{,}\PY{l+m+mi}{1}\PY{p}{)}\PY{p}{)}
        \PY{n}{plt}\PY{o}{.}\PY{n}{imshow}\PY{p}{(}\PY{n}{image}\PY{p}{,} \PY{n}{cmap}\PY{o}{=}\PY{l+s+s2}{\PYZdq{}}\PY{l+s+s2}{gray}\PY{l+s+s2}{\PYZdq{}}\PY{p}{)}
        \PY{n+nb}{print}\PY{p}{(}\PY{n}{y\PYZus{}train}\PY{p}{[}\PY{n}{index}\PY{p}{]}\PY{p}{)}
\end{Verbatim}

    \begin{Verbatim}[commandchars=\\\{\}]
17

    \end{Verbatim}

    \begin{center}
    \adjustimage{max size={0.9\linewidth}{0.9\paperheight}}{Traffic_Sign_Classifier_files/Traffic_Sign_Classifier_9_1.png}
    \end{center}
    { \hspace*{\fill} \\}
    
    \begin{center}\rule{0.5\linewidth}{\linethickness}\end{center}

\subsection{Step 2: Design and Test a Model
Architecture}\label{step-2-design-and-test-a-model-architecture}

Design and implement a deep learning model that learns to recognize
traffic signs. Train and test your model on the
\href{http://benchmark.ini.rub.de/?section=gtsrb\&subsection=dataset}{German
Traffic Sign Dataset}.

The LeNet-5 implementation shown in the
\href{https://classroom.udacity.com/nanodegrees/nd013/parts/fbf77062-5703-404e-b60c-95b78b2f3f9e/modules/6df7ae49-c61c-4bb2-a23e-6527e69209ec/lessons/601ae704-1035-4287-8b11-e2c2716217ad/concepts/d4aca031-508f-4e0b-b493-e7b706120f81}{classroom}
at the end of the CNN lesson is a solid starting point. You'll have to
change the number of classes and possibly the preprocessing, but aside
from that it's plug and play!

With the LeNet-5 solution from the lecture, you should expect a
validation set accuracy of about 0.89. To meet specifications, the
validation set accuracy will need to be at least 0.93. It is possible to
get an even higher accuracy, but 0.93 is the minimum for a successful
project submission.

There are various aspects to consider when thinking about this problem:

\begin{itemize}
\tightlist
\item
  Neural network architecture (is the network over or underfitting?)
\item
  Play around preprocessing techniques (normalization, rgb to grayscale,
  etc)
\item
  Number of examples per label (some have more than others).
\item
  Generate fake data.
\end{itemize}

Here is an example of a
\href{http://yann.lecun.com/exdb/publis/pdf/sermanet-ijcnn-11.pdf}{published
baseline model on this problem}. It's not required to be familiar with
the approach used in the paper but, it's good practice to try to read
papers like these.

    \subsubsection{Pre-process the Data Set (normalization, grayscale,
etc.)}\label{pre-process-the-data-set-normalization-grayscale-etc.}

    Minimally, the image data should be normalized so that the data has mean
zero and equal variance. For image data, \texttt{(pixel\ -\ 128)/\ 128}
is a quick way to approximately normalize the data and can be used in
this project.

Other pre-processing steps are optional. You can try different
techniques to see if it improves performance.

Use the code cell (or multiple code cells, if necessary) to implement
the first step of your project.

    \begin{Verbatim}[commandchars=\\\{\}]
{\color{incolor}In [{\color{incolor}5}]:} \PY{c+c1}{\PYZsh{}\PYZsh{}\PYZsh{} Preprocess the data here. It is required to normalize the data. Other preprocessing steps could include }
        \PY{c+c1}{\PYZsh{}\PYZsh{}\PYZsh{} converting to grayscale, etc.}
        \PY{c+c1}{\PYZsh{}\PYZsh{}\PYZsh{} Feel free to use as many code cells as needed.}
        
        \PY{k+kn}{from} \PY{n+nn}{sklearn}\PY{n+nn}{.}\PY{n+nn}{utils} \PY{k}{import} \PY{n}{shuffle}
        
        \PY{c+c1}{\PYZsh{} convert to grayscale; didn\PYZsq{}t improve performance}
        \PY{c+c1}{\PYZsh{}X\PYZus{}train[...,0] = np.dot(X\PYZus{}train[...,:3], [0.299, 0.587, 0.114])}
        \PY{c+c1}{\PYZsh{}X\PYZus{}valid[...,0] = np.dot(X\PYZus{}valid[...,:3], [0.299, 0.587, 0.114])}
        \PY{c+c1}{\PYZsh{}X\PYZus{}test[...,0] = np.dot(X\PYZus{}test[...,:3], [0.299, 0.587, 0.114])}
        \PY{c+c1}{\PYZsh{}X\PYZus{}train = np.delete(X\PYZus{}train, [1, 2], 3)}
        \PY{c+c1}{\PYZsh{}X\PYZus{}valid = np.delete(X\PYZus{}valid, [1, 2], 3)}
        \PY{c+c1}{\PYZsh{}X\PYZus{}test = np.delete(X\PYZus{}test, [1, 2], 3)}
        
        \PY{n}{X\PYZus{}train}\PY{p}{,} \PY{n}{y\PYZus{}train} \PY{o}{=} \PY{n}{shuffle}\PY{p}{(}\PY{n}{X\PYZus{}train}\PY{p}{,} \PY{n}{y\PYZus{}train}\PY{p}{)}
        
        \PY{c+c1}{\PYZsh{} normalize}
        \PY{n}{X\PYZus{}train} \PY{o}{=} \PY{p}{(}\PY{n}{X\PYZus{}train} \PY{o}{\PYZhy{}} \PY{n}{np}\PY{o}{.}\PY{n}{mean}\PY{p}{(}\PY{n}{X\PYZus{}train}\PY{p}{)}\PY{p}{)}\PY{o}{/}\PY{n}{np}\PY{o}{.}\PY{n}{std}\PY{p}{(}\PY{n}{X\PYZus{}train}\PY{p}{)}
        \PY{n}{X\PYZus{}valid} \PY{o}{=} \PY{p}{(}\PY{n}{X\PYZus{}valid} \PY{o}{\PYZhy{}} \PY{n}{np}\PY{o}{.}\PY{n}{mean}\PY{p}{(}\PY{n}{X\PYZus{}valid}\PY{p}{)}\PY{p}{)}\PY{o}{/}\PY{n}{np}\PY{o}{.}\PY{n}{std}\PY{p}{(}\PY{n}{X\PYZus{}valid}\PY{p}{)}
        \PY{n}{X\PYZus{}test} \PY{o}{=} \PY{p}{(}\PY{n}{X\PYZus{}test} \PY{o}{\PYZhy{}} \PY{n}{np}\PY{o}{.}\PY{n}{mean}\PY{p}{(}\PY{n}{X\PYZus{}test}\PY{p}{)}\PY{p}{)}\PY{o}{/}\PY{n}{np}\PY{o}{.}\PY{n}{std}\PY{p}{(}\PY{n}{X\PYZus{}test}\PY{p}{)}
\end{Verbatim}

    \subsubsection{Model Architecture}\label{model-architecture}

    \begin{Verbatim}[commandchars=\\\{\}]
{\color{incolor}In [{\color{incolor}6}]:} \PY{c+c1}{\PYZsh{}\PYZsh{}\PYZsh{} Define your architecture here.}
        \PY{c+c1}{\PYZsh{}\PYZsh{}\PYZsh{} Feel free to use as many code cells as needed.}
        \PY{k+kn}{from} \PY{n+nn}{tensorflow}\PY{n+nn}{.}\PY{n+nn}{contrib}\PY{n+nn}{.}\PY{n+nn}{layers} \PY{k}{import} \PY{n}{flatten}
        
        \PY{k}{def} \PY{n+nf}{LeNet}\PY{p}{(}\PY{n}{x}\PY{p}{)}\PY{p}{:}
            \PY{n}{C} \PY{o}{=} \PY{n+nb}{int}\PY{p}{(}\PY{n}{x}\PY{o}{.}\PY{n}{shape}\PY{p}{[}\PY{l+m+mi}{3}\PY{p}{]}\PY{p}{)}
            \PY{n}{conv1} \PY{o}{=} \PY{n}{tf}\PY{o}{.}\PY{n}{layers}\PY{o}{.}\PY{n}{conv2d}\PY{p}{(}\PY{n}{inputs}\PY{o}{=}\PY{n}{x}\PY{p}{,} \PY{n}{filters}\PY{o}{=}\PY{l+m+mi}{6}\PY{o}{*}\PY{n}{C}\PY{p}{,} \PY{n}{kernel\PYZus{}size}\PY{o}{=} \PY{l+m+mi}{5}\PY{p}{,} \PY{n}{padding}\PY{o}{=}\PY{l+s+s2}{\PYZdq{}}\PY{l+s+s2}{valid}\PY{l+s+s2}{\PYZdq{}}\PY{p}{,} \PY{n}{activation}\PY{o}{=}\PY{n}{tf}\PY{o}{.}\PY{n}{nn}\PY{o}{.}\PY{n}{relu}\PY{p}{)}
            \PY{n}{pool1} \PY{o}{=} \PY{n}{tf}\PY{o}{.}\PY{n}{layers}\PY{o}{.}\PY{n}{max\PYZus{}pooling2d}\PY{p}{(}\PY{n}{inputs}\PY{o}{=}\PY{n}{conv1}\PY{p}{,} \PY{n}{pool\PYZus{}size}\PY{o}{=}\PY{l+m+mi}{2}\PY{p}{,} \PY{n}{strides}\PY{o}{=}\PY{l+m+mi}{2}\PY{p}{)}
            \PY{n}{conv2} \PY{o}{=} \PY{n}{tf}\PY{o}{.}\PY{n}{layers}\PY{o}{.}\PY{n}{conv2d}\PY{p}{(}\PY{n}{inputs}\PY{o}{=}\PY{n}{pool1}\PY{p}{,} \PY{n}{filters}\PY{o}{=}\PY{l+m+mi}{16}\PY{o}{*}\PY{n}{C}\PY{p}{,} \PY{n}{kernel\PYZus{}size}\PY{o}{=}\PY{l+m+mi}{5}\PY{p}{,} \PY{n}{padding}\PY{o}{=}\PY{l+s+s2}{\PYZdq{}}\PY{l+s+s2}{valid}\PY{l+s+s2}{\PYZdq{}}\PY{p}{,} \PY{n}{activation}\PY{o}{=}\PY{n}{tf}\PY{o}{.}\PY{n}{nn}\PY{o}{.}\PY{n}{relu}\PY{p}{)}
            \PY{n}{pool2} \PY{o}{=} \PY{n}{tf}\PY{o}{.}\PY{n}{layers}\PY{o}{.}\PY{n}{max\PYZus{}pooling2d}\PY{p}{(}\PY{n}{inputs}\PY{o}{=}\PY{n}{conv2}\PY{p}{,} \PY{n}{pool\PYZus{}size}\PY{o}{=}\PY{l+m+mi}{2}\PY{p}{,} \PY{n}{strides}\PY{o}{=}\PY{l+m+mi}{2}\PY{p}{)}
            \PY{n}{flat2} \PY{o}{=} \PY{n}{flatten}\PY{p}{(}\PY{n}{pool2}\PY{p}{)}
            \PY{n}{dense3} \PY{o}{=} \PY{n}{tf}\PY{o}{.}\PY{n}{layers}\PY{o}{.}\PY{n}{dense}\PY{p}{(}\PY{n}{flat2}\PY{p}{,} \PY{l+m+mi}{120}\PY{o}{*}\PY{n}{C}\PY{p}{,} \PY{n}{activation}\PY{o}{=}\PY{n}{tf}\PY{o}{.}\PY{n}{nn}\PY{o}{.}\PY{n}{relu}\PY{p}{,} \PY{n}{use\PYZus{}bias}\PY{o}{=}\PY{k+kc}{True}\PY{p}{)}
            \PY{n}{dense4} \PY{o}{=} \PY{n}{tf}\PY{o}{.}\PY{n}{layers}\PY{o}{.}\PY{n}{dense}\PY{p}{(}\PY{n}{dense3}\PY{p}{,} \PY{l+m+mi}{84}\PY{o}{*}\PY{n}{C}\PY{p}{,} \PY{n}{activation}\PY{o}{=}\PY{n}{tf}\PY{o}{.}\PY{n}{nn}\PY{o}{.}\PY{n}{relu}\PY{p}{,} \PY{n}{use\PYZus{}bias}\PY{o}{=}\PY{k+kc}{True}\PY{p}{)}
            \PY{n}{logits} \PY{o}{=} \PY{n}{tf}\PY{o}{.}\PY{n}{layers}\PY{o}{.}\PY{n}{dense}\PY{p}{(}\PY{n}{dense4}\PY{p}{,} \PY{n}{n\PYZus{}classes}\PY{p}{,} \PY{n}{use\PYZus{}bias}\PY{o}{=}\PY{k+kc}{True}\PY{p}{)}
            
            \PY{k}{return} \PY{n}{logits}
        
        \PY{n}{x} \PY{o}{=} \PY{n}{tf}\PY{o}{.}\PY{n}{placeholder}\PY{p}{(}\PY{n}{tf}\PY{o}{.}\PY{n}{float32}\PY{p}{,} \PY{p}{(}\PY{k+kc}{None}\PY{p}{,} \PY{n}{X\PYZus{}train}\PY{o}{.}\PY{n}{shape}\PY{p}{[}\PY{l+m+mi}{1}\PY{p}{]}\PY{p}{,} \PY{n}{X\PYZus{}train}\PY{o}{.}\PY{n}{shape}\PY{p}{[}\PY{l+m+mi}{2}\PY{p}{]}\PY{p}{,} \PY{n}{X\PYZus{}train}\PY{o}{.}\PY{n}{shape}\PY{p}{[}\PY{l+m+mi}{3}\PY{p}{]}\PY{p}{)}\PY{p}{)}
        \PY{n}{y} \PY{o}{=} \PY{n}{tf}\PY{o}{.}\PY{n}{placeholder}\PY{p}{(}\PY{n}{tf}\PY{o}{.}\PY{n}{int32}\PY{p}{,} \PY{p}{(}\PY{k+kc}{None}\PY{p}{)}\PY{p}{)}
        \PY{n}{one\PYZus{}hot\PYZus{}y} \PY{o}{=} \PY{n}{tf}\PY{o}{.}\PY{n}{one\PYZus{}hot}\PY{p}{(}\PY{n}{y}\PY{p}{,} \PY{n}{n\PYZus{}classes}\PY{p}{)}
\end{Verbatim}

    \subsubsection{Train, Validate and Test the
Model}\label{train-validate-and-test-the-model}

    A validation set can be used to assess how well the model is performing.
A low accuracy on the training and validation sets imply underfitting. A
high accuracy on the training set but low accuracy on the validation set
implies overfitting.

    \begin{Verbatim}[commandchars=\\\{\}]
{\color{incolor}In [{\color{incolor}15}]:} \PY{c+c1}{\PYZsh{}\PYZsh{}\PYZsh{} Train your model here.}
         \PY{c+c1}{\PYZsh{}\PYZsh{}\PYZsh{} Calculate and report the accuracy on the training and validation set.}
         \PY{c+c1}{\PYZsh{}\PYZsh{}\PYZsh{} Once a final model architecture is selected, }
         \PY{c+c1}{\PYZsh{}\PYZsh{}\PYZsh{} the accuracy on the test set should be calculated and reported as well.}
         \PY{c+c1}{\PYZsh{}\PYZsh{}\PYZsh{} Feel free to use as many code cells as needed.}
         \PY{n}{rate} \PY{o}{=} \PY{l+m+mf}{0.001}
         
         \PY{n}{logits} \PY{o}{=} \PY{n}{LeNet}\PY{p}{(}\PY{n}{x}\PY{p}{)}
         \PY{n}{cross\PYZus{}entropy} \PY{o}{=} \PY{n}{tf}\PY{o}{.}\PY{n}{nn}\PY{o}{.}\PY{n}{softmax\PYZus{}cross\PYZus{}entropy\PYZus{}with\PYZus{}logits}\PY{p}{(}\PY{n}{labels}\PY{o}{=}\PY{n}{one\PYZus{}hot\PYZus{}y}\PY{p}{,} \PY{n}{logits}\PY{o}{=}\PY{n}{logits}\PY{p}{)}
         \PY{n}{loss\PYZus{}operation} \PY{o}{=} \PY{n}{tf}\PY{o}{.}\PY{n}{reduce\PYZus{}mean}\PY{p}{(}\PY{n}{cross\PYZus{}entropy}\PY{p}{)}
         \PY{n}{optimizer} \PY{o}{=} \PY{n}{tf}\PY{o}{.}\PY{n}{train}\PY{o}{.}\PY{n}{AdamOptimizer}\PY{p}{(}\PY{n}{learning\PYZus{}rate} \PY{o}{=} \PY{n}{rate}\PY{p}{)}
         \PY{n}{training\PYZus{}operation} \PY{o}{=} \PY{n}{optimizer}\PY{o}{.}\PY{n}{minimize}\PY{p}{(}\PY{n}{loss\PYZus{}operation}\PY{p}{)}
         
         \PY{n}{correct\PYZus{}prediction} \PY{o}{=} \PY{n}{tf}\PY{o}{.}\PY{n}{equal}\PY{p}{(}\PY{n}{tf}\PY{o}{.}\PY{n}{argmax}\PY{p}{(}\PY{n}{logits}\PY{p}{,} \PY{l+m+mi}{1}\PY{p}{)}\PY{p}{,} \PY{n}{tf}\PY{o}{.}\PY{n}{argmax}\PY{p}{(}\PY{n}{one\PYZus{}hot\PYZus{}y}\PY{p}{,} \PY{l+m+mi}{1}\PY{p}{)}\PY{p}{)}
         \PY{n}{accuracy\PYZus{}operation} \PY{o}{=} \PY{n}{tf}\PY{o}{.}\PY{n}{reduce\PYZus{}mean}\PY{p}{(}\PY{n}{tf}\PY{o}{.}\PY{n}{cast}\PY{p}{(}\PY{n}{correct\PYZus{}prediction}\PY{p}{,} \PY{n}{tf}\PY{o}{.}\PY{n}{float32}\PY{p}{)}\PY{p}{)}
         
         \PY{n}{saver} \PY{o}{=} \PY{n}{tf}\PY{o}{.}\PY{n}{train}\PY{o}{.}\PY{n}{Saver}\PY{p}{(}\PY{p}{)}
         
         \PY{k}{def} \PY{n+nf}{evaluate}\PY{p}{(}\PY{n}{X\PYZus{}data}\PY{p}{,} \PY{n}{y\PYZus{}data}\PY{p}{)}\PY{p}{:}
             \PY{n}{num\PYZus{}examples} \PY{o}{=} \PY{n+nb}{len}\PY{p}{(}\PY{n}{X\PYZus{}data}\PY{p}{)}
             \PY{n}{total\PYZus{}accuracy} \PY{o}{=} \PY{l+m+mi}{0}
             \PY{n}{sess} \PY{o}{=} \PY{n}{tf}\PY{o}{.}\PY{n}{get\PYZus{}default\PYZus{}session}\PY{p}{(}\PY{p}{)}
             \PY{k}{for} \PY{n}{offset} \PY{o+ow}{in} \PY{n+nb}{range}\PY{p}{(}\PY{l+m+mi}{0}\PY{p}{,} \PY{n}{num\PYZus{}examples}\PY{p}{,} \PY{n}{BATCH\PYZus{}SIZE}\PY{p}{)}\PY{p}{:}
                 \PY{n}{batch\PYZus{}x}\PY{p}{,} \PY{n}{batch\PYZus{}y} \PY{o}{=} \PY{n}{X\PYZus{}data}\PY{p}{[}\PY{n}{offset}\PY{p}{:}\PY{n}{offset}\PY{o}{+}\PY{n}{BATCH\PYZus{}SIZE}\PY{p}{]}\PY{p}{,} \PY{n}{y\PYZus{}data}\PY{p}{[}\PY{n}{offset}\PY{p}{:}\PY{n}{offset}\PY{o}{+}\PY{n}{BATCH\PYZus{}SIZE}\PY{p}{]}
                 \PY{n}{accuracy} \PY{o}{=} \PY{n}{sess}\PY{o}{.}\PY{n}{run}\PY{p}{(}\PY{n}{accuracy\PYZus{}operation}\PY{p}{,} \PY{n}{feed\PYZus{}dict}\PY{o}{=}\PY{p}{\PYZob{}}\PY{n}{x}\PY{p}{:} \PY{n}{batch\PYZus{}x}\PY{p}{,} \PY{n}{y}\PY{p}{:} \PY{n}{batch\PYZus{}y}\PY{p}{\PYZcb{}}\PY{p}{)}
                 \PY{n}{total\PYZus{}accuracy} \PY{o}{+}\PY{o}{=} \PY{p}{(}\PY{n}{accuracy} \PY{o}{*} \PY{n+nb}{len}\PY{p}{(}\PY{n}{batch\PYZus{}x}\PY{p}{)}\PY{p}{)}
             \PY{k}{return} \PY{n}{total\PYZus{}accuracy} \PY{o}{/} \PY{n}{num\PYZus{}examples}
         
         \PY{k}{with} \PY{n}{tf}\PY{o}{.}\PY{n}{Session}\PY{p}{(}\PY{p}{)} \PY{k}{as} \PY{n}{sess}\PY{p}{:}
             \PY{n}{sess}\PY{o}{.}\PY{n}{run}\PY{p}{(}\PY{n}{tf}\PY{o}{.}\PY{n}{global\PYZus{}variables\PYZus{}initializer}\PY{p}{(}\PY{p}{)}\PY{p}{)}
             \PY{n}{num\PYZus{}examples} \PY{o}{=} \PY{n+nb}{len}\PY{p}{(}\PY{n}{X\PYZus{}train}\PY{p}{)}
             
             \PY{n+nb}{print}\PY{p}{(}\PY{l+s+s2}{\PYZdq{}}\PY{l+s+s2}{Training...}\PY{l+s+s2}{\PYZdq{}}\PY{p}{)}
             \PY{n+nb}{print}\PY{p}{(}\PY{p}{)}
             \PY{k}{for} \PY{n}{i} \PY{o+ow}{in} \PY{n+nb}{range}\PY{p}{(}\PY{n}{EPOCHS}\PY{p}{)}\PY{p}{:}
                 \PY{n}{X\PYZus{}train}\PY{p}{,} \PY{n}{y\PYZus{}train} \PY{o}{=} \PY{n}{shuffle}\PY{p}{(}\PY{n}{X\PYZus{}train}\PY{p}{,} \PY{n}{y\PYZus{}train}\PY{p}{)}
                 \PY{k}{for} \PY{n}{offset} \PY{o+ow}{in} \PY{n+nb}{range}\PY{p}{(}\PY{l+m+mi}{0}\PY{p}{,} \PY{n}{num\PYZus{}examples}\PY{p}{,} \PY{n}{BATCH\PYZus{}SIZE}\PY{p}{)}\PY{p}{:}
                     \PY{n}{end} \PY{o}{=} \PY{n}{offset} \PY{o}{+} \PY{n}{BATCH\PYZus{}SIZE}
                     \PY{n}{batch\PYZus{}x}\PY{p}{,} \PY{n}{batch\PYZus{}y} \PY{o}{=} \PY{n}{X\PYZus{}train}\PY{p}{[}\PY{n}{offset}\PY{p}{:}\PY{n}{end}\PY{p}{]}\PY{p}{,} \PY{n}{y\PYZus{}train}\PY{p}{[}\PY{n}{offset}\PY{p}{:}\PY{n}{end}\PY{p}{]}
                     \PY{n}{sess}\PY{o}{.}\PY{n}{run}\PY{p}{(}\PY{n}{training\PYZus{}operation}\PY{p}{,} \PY{n}{feed\PYZus{}dict}\PY{o}{=}\PY{p}{\PYZob{}}\PY{n}{x}\PY{p}{:} \PY{n}{batch\PYZus{}x}\PY{p}{,} \PY{n}{y}\PY{p}{:} \PY{n}{batch\PYZus{}y}\PY{p}{\PYZcb{}}\PY{p}{)}
                     
                 \PY{n}{validation\PYZus{}accuracy} \PY{o}{=} \PY{n}{evaluate}\PY{p}{(}\PY{n}{X\PYZus{}valid}\PY{p}{,} \PY{n}{y\PYZus{}valid}\PY{p}{)}
                 \PY{n+nb}{print}\PY{p}{(}\PY{l+s+s2}{\PYZdq{}}\PY{l+s+s2}{EPOCH }\PY{l+s+si}{\PYZob{}\PYZcb{}}\PY{l+s+s2}{ ...}\PY{l+s+s2}{\PYZdq{}}\PY{o}{.}\PY{n}{format}\PY{p}{(}\PY{n}{i}\PY{o}{+}\PY{l+m+mi}{1}\PY{p}{)}\PY{p}{)}
                 \PY{n+nb}{print}\PY{p}{(}\PY{l+s+s2}{\PYZdq{}}\PY{l+s+s2}{Validation Accuracy = }\PY{l+s+si}{\PYZob{}:.3f\PYZcb{}}\PY{l+s+s2}{\PYZdq{}}\PY{o}{.}\PY{n}{format}\PY{p}{(}\PY{n}{validation\PYZus{}accuracy}\PY{p}{)}\PY{p}{)}
                 \PY{n+nb}{print}\PY{p}{(}\PY{p}{)}
                 
             \PY{n}{saver}\PY{o}{.}\PY{n}{save}\PY{p}{(}\PY{n}{sess}\PY{p}{,} \PY{l+s+s1}{\PYZsq{}}\PY{l+s+s1}{./lenet}\PY{l+s+s1}{\PYZsq{}}\PY{p}{)}
             \PY{n+nb}{print}\PY{p}{(}\PY{l+s+s2}{\PYZdq{}}\PY{l+s+s2}{Model saved}\PY{l+s+s2}{\PYZdq{}}\PY{p}{)}
             
             \PY{k}{with} \PY{n}{tf}\PY{o}{.}\PY{n}{Session}\PY{p}{(}\PY{p}{)} \PY{k}{as} \PY{n}{sess}\PY{p}{:}
                 \PY{n}{saver}\PY{o}{.}\PY{n}{restore}\PY{p}{(}\PY{n}{sess}\PY{p}{,} \PY{n}{tf}\PY{o}{.}\PY{n}{train}\PY{o}{.}\PY{n}{latest\PYZus{}checkpoint}\PY{p}{(}\PY{l+s+s1}{\PYZsq{}}\PY{l+s+s1}{.}\PY{l+s+s1}{\PYZsq{}}\PY{p}{)}\PY{p}{)}
         
             \PY{n}{test\PYZus{}accuracy} \PY{o}{=} \PY{n}{evaluate}\PY{p}{(}\PY{n}{X\PYZus{}test}\PY{p}{,} \PY{n}{y\PYZus{}test}\PY{p}{)}
             \PY{n+nb}{print}\PY{p}{(}\PY{l+s+s2}{\PYZdq{}}\PY{l+s+s2}{Test Accuracy = }\PY{l+s+si}{\PYZob{}:.3f\PYZcb{}}\PY{l+s+s2}{\PYZdq{}}\PY{o}{.}\PY{n}{format}\PY{p}{(}\PY{n}{test\PYZus{}accuracy}\PY{p}{)}\PY{p}{)}
\end{Verbatim}

    \begin{Verbatim}[commandchars=\\\{\}]
Training{\ldots}

EPOCH 1 {\ldots}
Validation Accuracy = 0.896

EPOCH 2 {\ldots}
Validation Accuracy = 0.922

EPOCH 3 {\ldots}
Validation Accuracy = 0.929

EPOCH 4 {\ldots}
Validation Accuracy = 0.917

EPOCH 5 {\ldots}
Validation Accuracy = 0.903

EPOCH 6 {\ldots}
Validation Accuracy = 0.946

EPOCH 7 {\ldots}
Validation Accuracy = 0.930

EPOCH 8 {\ldots}
Validation Accuracy = 0.951

EPOCH 9 {\ldots}
Validation Accuracy = 0.951

EPOCH 10 {\ldots}
Validation Accuracy = 0.957

Model saved
Test Accuracy = 0.942

    \end{Verbatim}

    \begin{center}\rule{0.5\linewidth}{\linethickness}\end{center}

\subsection{Step 3: Test a Model on New
Images}\label{step-3-test-a-model-on-new-images}

To give yourself more insight into how your model is working, download
at least five pictures of German traffic signs from the web and use your
model to predict the traffic sign type.

You may find \texttt{signnames.csv} useful as it contains mappings from
the class id (integer) to the actual sign name.

    \subsubsection{Load and Output the
Images}\label{load-and-output-the-images}

    \begin{Verbatim}[commandchars=\\\{\}]
{\color{incolor}In [{\color{incolor}16}]:} \PY{c+c1}{\PYZsh{}\PYZsh{}\PYZsh{} Load the images and plot them here.}
         \PY{c+c1}{\PYZsh{}\PYZsh{}\PYZsh{} Feel free to use as many code cells as needed.}
         
         \PY{n}{predict\PYZus{}file} \PY{o}{=} \PY{l+s+s1}{\PYZsq{}}\PY{l+s+s1}{dataset/predict.p}\PY{l+s+s1}{\PYZsq{}}
         
         \PY{k}{with} \PY{n+nb}{open}\PY{p}{(}\PY{n}{predict\PYZus{}file}\PY{p}{,} \PY{n}{mode}\PY{o}{=}\PY{l+s+s1}{\PYZsq{}}\PY{l+s+s1}{rb}\PY{l+s+s1}{\PYZsq{}}\PY{p}{)} \PY{k}{as} \PY{n}{f}\PY{p}{:}
             \PY{n}{predict} \PY{o}{=} \PY{n}{pickle}\PY{o}{.}\PY{n}{load}\PY{p}{(}\PY{n}{f}\PY{p}{)}
             
         \PY{n}{X\PYZus{}predict}\PY{p}{,} \PY{n}{y\PYZus{}predict} \PY{o}{=} \PY{n}{predict}\PY{p}{[}\PY{l+s+s1}{\PYZsq{}}\PY{l+s+s1}{features}\PY{l+s+s1}{\PYZsq{}}\PY{p}{]}\PY{p}{,} \PY{n}{predict}\PY{p}{[}\PY{l+s+s1}{\PYZsq{}}\PY{l+s+s1}{labels}\PY{l+s+s1}{\PYZsq{}}\PY{p}{]}
         
         \PY{k}{assert}\PY{p}{(}\PY{n+nb}{len}\PY{p}{(}\PY{n}{X\PYZus{}predict}\PY{p}{)} \PY{o}{==} \PY{n+nb}{len}\PY{p}{(}\PY{n}{y\PYZus{}predict}\PY{p}{)}\PY{p}{)}
         
         \PY{n}{n\PYZus{}predict} \PY{o}{=} \PY{n+nb}{len}\PY{p}{(}\PY{n}{X\PYZus{}predict}\PY{p}{)}
         \PY{n}{image\PYZus{}shape} \PY{o}{=} \PY{n}{X\PYZus{}predict}\PY{p}{[}\PY{l+m+mi}{0}\PY{p}{]}\PY{o}{.}\PY{n}{shape}
         \PY{n+nb}{print}\PY{p}{(}\PY{l+s+s2}{\PYZdq{}}\PY{l+s+s2}{Number of prediction examples =}\PY{l+s+s2}{\PYZdq{}}\PY{p}{,} \PY{n}{n\PYZus{}predict}\PY{p}{)}
         \PY{n+nb}{print}\PY{p}{(}\PY{l+s+s2}{\PYZdq{}}\PY{l+s+s2}{Image shape: }\PY{l+s+s2}{\PYZdq{}}\PY{p}{,} \PY{n}{image\PYZus{}shape}\PY{p}{)}
\end{Verbatim}

    \begin{Verbatim}[commandchars=\\\{\}]
Number of prediction examples = 5
Image shape:  (32, 32, 3)

    \end{Verbatim}

    \begin{Verbatim}[commandchars=\\\{\}]
{\color{incolor}In [{\color{incolor}17}]:} \PY{n}{index} \PY{o}{=} \PY{n}{random}\PY{o}{.}\PY{n}{randint}\PY{p}{(}\PY{l+m+mi}{0}\PY{p}{,} \PY{n+nb}{len}\PY{p}{(}\PY{n}{X\PYZus{}predict}\PY{p}{)}\PY{o}{\PYZhy{}}\PY{l+m+mi}{1}\PY{p}{)}
         \PY{n}{image} \PY{o}{=} \PY{n}{X\PYZus{}predict}\PY{p}{[}\PY{n}{index}\PY{p}{]}
         
         \PY{c+c1}{\PYZsh{} Visualizations will be shown in the notebook.}
         \PY{o}{\PYZpc{}}\PY{k}{matplotlib} inline
         
         \PY{n}{plt}\PY{o}{.}\PY{n}{figure}\PY{p}{(}\PY{n}{figsize}\PY{o}{=}\PY{p}{(}\PY{l+m+mi}{1}\PY{p}{,}\PY{l+m+mi}{1}\PY{p}{)}\PY{p}{)}
         \PY{n}{plt}\PY{o}{.}\PY{n}{imshow}\PY{p}{(}\PY{n}{image}\PY{p}{)}
         \PY{n+nb}{print}\PY{p}{(}\PY{n}{y\PYZus{}predict}\PY{p}{[}\PY{n}{index}\PY{p}{]}\PY{p}{)}
\end{Verbatim}

    \begin{Verbatim}[commandchars=\\\{\}]
01

    \end{Verbatim}

    \begin{center}
    \adjustimage{max size={0.9\linewidth}{0.9\paperheight}}{Traffic_Sign_Classifier_files/Traffic_Sign_Classifier_22_1.png}
    \end{center}
    { \hspace*{\fill} \\}
    
    \begin{Verbatim}[commandchars=\\\{\}]
{\color{incolor}In [{\color{incolor}18}]:} \PY{c+c1}{\PYZsh{}\PYZsh{}\PYZsh{} Run the predictions here and use the model to output the prediction for each image.}
         \PY{c+c1}{\PYZsh{}\PYZsh{}\PYZsh{} Make sure to pre\PYZhy{}process the images with the same pre\PYZhy{}processing pipeline used earlier.}
         \PY{c+c1}{\PYZsh{}\PYZsh{}\PYZsh{} Feel free to use as many code cells as needed.}
         
         \PY{k+kn}{import} \PY{n+nn}{csv}
         
         \PY{k}{with} \PY{n+nb}{open}\PY{p}{(}\PY{l+s+s1}{\PYZsq{}}\PY{l+s+s1}{signnames.csv}\PY{l+s+s1}{\PYZsq{}}\PY{p}{,} \PY{l+s+s1}{\PYZsq{}}\PY{l+s+s1}{r}\PY{l+s+s1}{\PYZsq{}}\PY{p}{)} \PY{k}{as} \PY{n}{f}\PY{p}{:}
             \PY{n}{reader} \PY{o}{=} \PY{n}{csv}\PY{o}{.}\PY{n}{reader}\PY{p}{(}\PY{n}{f}\PY{p}{)}
             \PY{n+nb}{next}\PY{p}{(}\PY{n}{reader}\PY{p}{)}
             \PY{n}{labels} \PY{o}{=} \PY{p}{\PYZob{}}\PY{p}{\PYZcb{}}
             \PY{k}{for} \PY{n}{row} \PY{o+ow}{in} \PY{n}{reader}\PY{p}{:}
                 \PY{n}{label} \PY{o}{=} \PY{n+nb}{int}\PY{p}{(}\PY{n}{row}\PY{p}{[}\PY{l+m+mi}{0}\PY{p}{]}\PY{p}{)}
                 \PY{n}{description} \PY{o}{=} \PY{n}{row}\PY{p}{[}\PY{l+m+mi}{1}\PY{p}{]}
                 \PY{n}{labels}\PY{o}{.}\PY{n}{update}\PY{p}{(}\PY{p}{\PYZob{}}\PY{n}{label}\PY{p}{:} \PY{n}{description}\PY{p}{\PYZcb{}}\PY{p}{)}
         \PY{n}{X\PYZus{}predict}\PY{p}{,} \PY{n}{y\PYZus{}predict} \PY{o}{=} \PY{n}{shuffle}\PY{p}{(}\PY{n}{X\PYZus{}predict}\PY{p}{,} \PY{n}{y\PYZus{}predict}\PY{p}{)}
         
         \PY{c+c1}{\PYZsh{} normalize}
         \PY{n}{X\PYZus{}predict} \PY{o}{=} \PY{p}{(}\PY{n}{X\PYZus{}predict} \PY{o}{\PYZhy{}} \PY{n}{np}\PY{o}{.}\PY{n}{mean}\PY{p}{(}\PY{n}{X\PYZus{}predict}\PY{p}{)}\PY{p}{)}\PY{o}{/}\PY{n}{np}\PY{o}{.}\PY{n}{std}\PY{p}{(}\PY{n}{X\PYZus{}predict}\PY{p}{)}
         
         \PY{n}{prediction\PYZus{}operation} \PY{o}{=} \PY{n}{tf}\PY{o}{.}\PY{n}{argmax}\PY{p}{(}\PY{n}{logits}\PY{p}{,} \PY{l+m+mi}{1}\PY{p}{)}
         \PY{n}{top5\PYZus{}operation} \PY{o}{=} \PY{n}{tf}\PY{o}{.}\PY{n}{nn}\PY{o}{.}\PY{n}{top\PYZus{}k}\PY{p}{(}\PY{n}{logits}\PY{p}{,} \PY{n}{k}\PY{o}{=}\PY{l+m+mi}{5}\PY{p}{)}
         
         \PY{k}{def} \PY{n+nf}{predict}\PY{p}{(}\PY{n}{X\PYZus{}data}\PY{p}{,} \PY{n}{y\PYZus{}data}\PY{p}{)}\PY{p}{:}
             \PY{n}{num\PYZus{}examples} \PY{o}{=} \PY{n+nb}{len}\PY{p}{(}\PY{n}{X\PYZus{}data}\PY{p}{)}
             \PY{n}{sess} \PY{o}{=} \PY{n}{tf}\PY{o}{.}\PY{n}{get\PYZus{}default\PYZus{}session}\PY{p}{(}\PY{p}{)}
             \PY{n}{prediction} \PY{o}{=} \PY{p}{[}\PY{p}{]}
             \PY{n}{top5} \PY{o}{=} \PY{p}{[}\PY{p}{]}
             \PY{k}{for} \PY{n}{offset} \PY{o+ow}{in} \PY{n+nb}{range}\PY{p}{(}\PY{l+m+mi}{0}\PY{p}{,} \PY{n}{num\PYZus{}examples}\PY{p}{,} \PY{n}{BATCH\PYZus{}SIZE}\PY{p}{)}\PY{p}{:}
                 \PY{n}{batch\PYZus{}x}\PY{p}{,} \PY{n}{batch\PYZus{}y} \PY{o}{=} \PY{n}{X\PYZus{}data}\PY{p}{[}\PY{n}{offset}\PY{p}{:}\PY{n}{offset}\PY{o}{+}\PY{n}{BATCH\PYZus{}SIZE}\PY{p}{]}\PY{p}{,} \PY{n}{y\PYZus{}data}\PY{p}{[}\PY{n}{offset}\PY{p}{:}\PY{n}{offset}\PY{o}{+}\PY{n}{BATCH\PYZus{}SIZE}\PY{p}{]}
                 \PY{n}{prediction}\PY{o}{.}\PY{n}{append}\PY{p}{(}\PY{n}{sess}\PY{o}{.}\PY{n}{run}\PY{p}{(}\PY{n}{prediction\PYZus{}operation}\PY{p}{,} \PY{n}{feed\PYZus{}dict}\PY{o}{=}\PY{p}{\PYZob{}}\PY{n}{x}\PY{p}{:} \PY{n}{batch\PYZus{}x}\PY{p}{,} \PY{n}{y}\PY{p}{:} \PY{n}{batch\PYZus{}y}\PY{p}{\PYZcb{}}\PY{p}{)}\PY{p}{)}
                 \PY{n}{top5}\PY{o}{.}\PY{n}{append}\PY{p}{(}\PY{n}{sess}\PY{o}{.}\PY{n}{run}\PY{p}{(}\PY{n}{top5\PYZus{}operation}\PY{p}{,} \PY{n}{feed\PYZus{}dict}\PY{o}{=}\PY{p}{\PYZob{}}\PY{n}{x}\PY{p}{:} \PY{n}{batch\PYZus{}x}\PY{p}{,} \PY{n}{y}\PY{p}{:} \PY{n}{batch\PYZus{}y}\PY{p}{\PYZcb{}}\PY{p}{)}\PY{p}{)}
             \PY{k}{return} \PY{n}{np}\PY{o}{.}\PY{n}{array}\PY{p}{(}\PY{n}{prediction}\PY{p}{[}\PY{l+m+mi}{0}\PY{p}{]}\PY{p}{)}\PY{p}{,} \PY{n}{top5}\PY{p}{[}\PY{l+m+mi}{0}\PY{p}{]}
         
         \PY{k}{with} \PY{n}{tf}\PY{o}{.}\PY{n}{Session}\PY{p}{(}\PY{p}{)} \PY{k}{as} \PY{n}{sess}\PY{p}{:}
             \PY{n}{saver}\PY{o}{.}\PY{n}{restore}\PY{p}{(}\PY{n}{sess}\PY{p}{,} \PY{n}{tf}\PY{o}{.}\PY{n}{train}\PY{o}{.}\PY{n}{latest\PYZus{}checkpoint}\PY{p}{(}\PY{l+s+s1}{\PYZsq{}}\PY{l+s+s1}{.}\PY{l+s+s1}{\PYZsq{}}\PY{p}{)}\PY{p}{)}
             \PY{p}{[}\PY{n}{prediction}\PY{p}{,} \PY{n}{top5}\PY{p}{]} \PY{o}{=} \PY{n}{predict}\PY{p}{(}\PY{n}{X\PYZus{}predict}\PY{p}{,} \PY{n}{y\PYZus{}predict}\PY{p}{)}
             \PY{c+c1}{\PYZsh{} Visualizations will be shown in the notebook.}
             \PY{o}{\PYZpc{}}\PY{k}{matplotlib} inline
             \PY{n}{fig} \PY{o}{=} \PY{n}{plt}\PY{o}{.}\PY{n}{figure}\PY{p}{(}\PY{p}{)}
             \PY{n+nb}{print}\PY{p}{(}\PY{l+s+s2}{\PYZdq{}}\PY{l+s+s2}{Predictions:}\PY{l+s+s2}{\PYZdq{}}\PY{p}{)}
             \PY{k}{for} \PY{n}{i} \PY{o+ow}{in} \PY{n+nb}{range}\PY{p}{(}\PY{n}{n\PYZus{}predict}\PY{p}{)}\PY{p}{:}
                 \PY{n}{image} \PY{o}{=} \PY{n}{X\PYZus{}predict}\PY{p}{[}\PY{n}{i}\PY{p}{]}
                 \PY{n}{ax} \PY{o}{=} \PY{n}{fig}\PY{o}{.}\PY{n}{add\PYZus{}subplot}\PY{p}{(}\PY{n+nb}{str}\PY{p}{(}\PY{l+m+mi}{151}\PY{o}{+}\PY{n}{i}\PY{p}{)}\PY{p}{)}
                 \PY{n}{txt} \PY{o}{=} \PY{l+s+s1}{\PYZsq{}}\PY{l+s+s1}{Image }\PY{l+s+si}{\PYZob{}\PYZcb{}}\PY{l+s+s1}{ \PYZhy{} }\PY{l+s+s1}{\PYZsq{}}\PY{o}{.}\PY{n}{format}\PY{p}{(}\PY{n}{i}\PY{o}{+}\PY{l+m+mi}{1}\PY{p}{)}\PY{o}{+}\PY{n}{labels}\PY{p}{[}\PY{n}{prediction}\PY{p}{[}\PY{n}{i}\PY{p}{]}\PY{p}{]}
                 \PY{n+nb}{print}\PY{p}{(}\PY{n}{txt}\PY{p}{)}
                 \PY{n}{ax}\PY{o}{.}\PY{n}{set\PYZus{}title}\PY{p}{(}\PY{l+s+s2}{\PYZdq{}}\PY{l+s+s2}{Image }\PY{l+s+s2}{\PYZdq{}}\PY{o}{+}\PY{n+nb}{str}\PY{p}{(}\PY{n}{i}\PY{o}{+}\PY{l+m+mi}{1}\PY{p}{)}\PY{p}{)}
                 \PY{n}{plt}\PY{o}{.}\PY{n}{imshow}\PY{p}{(}\PY{n}{image}\PY{p}{)}
\end{Verbatim}

    \begin{Verbatim}[commandchars=\\\{\}]
Predictions:
Image 1 - Go straight or right
Image 2 - No passing for vehicles over 3.5 metric tons
Image 3 - Yield
Image 4 - Speed limit (30km/h)
Image 5 - General caution

    \end{Verbatim}

    \begin{center}
    \adjustimage{max size={0.9\linewidth}{0.9\paperheight}}{Traffic_Sign_Classifier_files/Traffic_Sign_Classifier_23_1.png}
    \end{center}
    { \hspace*{\fill} \\}
    
    \subsubsection{Predict the Sign Type for Each
Image}\label{predict-the-sign-type-for-each-image}

    \subsubsection{Analyze Performance}\label{analyze-performance}

    \begin{Verbatim}[commandchars=\\\{\}]
{\color{incolor}In [{\color{incolor}19}]:} \PY{c+c1}{\PYZsh{}\PYZsh{}\PYZsh{} Calculate the accuracy for these 5 new images. }
         \PY{c+c1}{\PYZsh{}\PYZsh{}\PYZsh{} For example, if the model predicted 1 out of 5 signs correctly, it\PYZsq{}s 20\PYZpc{} accurate on these new images.}
         
         \PY{k}{with} \PY{n}{tf}\PY{o}{.}\PY{n}{Session}\PY{p}{(}\PY{p}{)} \PY{k}{as} \PY{n}{sess}\PY{p}{:}
             \PY{n}{saver}\PY{o}{.}\PY{n}{restore}\PY{p}{(}\PY{n}{sess}\PY{p}{,} \PY{n}{tf}\PY{o}{.}\PY{n}{train}\PY{o}{.}\PY{n}{latest\PYZus{}checkpoint}\PY{p}{(}\PY{l+s+s1}{\PYZsq{}}\PY{l+s+s1}{.}\PY{l+s+s1}{\PYZsq{}}\PY{p}{)}\PY{p}{)}
             \PY{n}{predict\PYZus{}accuracy} \PY{o}{=} \PY{n}{evaluate}\PY{p}{(}\PY{n}{X\PYZus{}predict}\PY{p}{,} \PY{n}{y\PYZus{}predict}\PY{p}{)}
             \PY{n+nb}{print}\PY{p}{(}\PY{l+s+s2}{\PYZdq{}}\PY{l+s+s2}{Test Accuracy = }\PY{l+s+si}{\PYZob{}:.0f\PYZcb{}}\PY{l+s+s2}{\PYZpc{}}\PY{l+s+s2}{\PYZdq{}}\PY{o}{.}\PY{n}{format}\PY{p}{(}\PY{n}{predict\PYZus{}accuracy}\PY{o}{*}\PY{l+m+mi}{100}\PY{p}{)}\PY{p}{)}
\end{Verbatim}

    \begin{Verbatim}[commandchars=\\\{\}]
Test Accuracy = 100\%

    \end{Verbatim}

    \subsubsection{Output Top 5 Softmax Probabilities For Each Image Found
on the
Web}\label{output-top-5-softmax-probabilities-for-each-image-found-on-the-web}

    For each of the new images, print out the model's softmax probabilities
to show the \textbf{certainty} of the model's predictions (limit the
output to the top 5 probabilities for each image).
\href{https://www.tensorflow.org/versions/r0.12/api_docs/python/nn.html\#top_k}{\texttt{tf.nn.top\_k}}
could prove helpful here.

The example below demonstrates how tf.nn.top\_k can be used to find the
top k predictions for each image.

\texttt{tf.nn.top\_k} will return the values and indices (class ids) of
the top k predictions. So if k=3, for each sign, it'll return the 3
largest probabilities (out of a possible 43) and the correspoding class
ids.

Take this numpy array as an example. The values in the array represent
predictions. The array contains softmax probabilities for five candidate
images with six possible classes. \texttt{tk.nn.top\_k} is used to
choose the three classes with the highest probability:

\begin{verbatim}
# (5, 6) array
a = np.array([[ 0.24879643,  0.07032244,  0.12641572,  0.34763842,  0.07893497,
         0.12789202],
       [ 0.28086119,  0.27569815,  0.08594638,  0.0178669 ,  0.18063401,
         0.15899337],
       [ 0.26076848,  0.23664738,  0.08020603,  0.07001922,  0.1134371 ,
         0.23892179],
       [ 0.11943333,  0.29198961,  0.02605103,  0.26234032,  0.1351348 ,
         0.16505091],
       [ 0.09561176,  0.34396535,  0.0643941 ,  0.16240774,  0.24206137,
         0.09155967]])
\end{verbatim}

Running it through \texttt{sess.run(tf.nn.top\_k(tf.constant(a),\ k=3))}
produces:

\begin{verbatim}
TopKV2(values=array([[ 0.34763842,  0.24879643,  0.12789202],
       [ 0.28086119,  0.27569815,  0.18063401],
       [ 0.26076848,  0.23892179,  0.23664738],
       [ 0.29198961,  0.26234032,  0.16505091],
       [ 0.34396535,  0.24206137,  0.16240774]]), indices=array([[3, 0, 5],
       [0, 1, 4],
       [0, 5, 1],
       [1, 3, 5],
       [1, 4, 3]], dtype=int32))
\end{verbatim}

Looking just at the first row we get
\texttt{{[}\ 0.34763842,\ \ 0.24879643,\ \ 0.12789202{]}}, you can
confirm these are the 3 largest probabilities in \texttt{a}. You'll also
notice \texttt{{[}3,\ 0,\ 5{]}} are the corresponding indices.

    \begin{Verbatim}[commandchars=\\\{\}]
{\color{incolor}In [{\color{incolor}20}]:} \PY{c+c1}{\PYZsh{}\PYZsh{}\PYZsh{} Print out the top five softmax probabilities for the predictions on the German traffic sign images found on the web. }
         \PY{c+c1}{\PYZsh{}\PYZsh{}\PYZsh{} Feel free to use as many code cells as needed.}
         \PY{n}{softmax} \PY{o}{=} \PY{n}{top5}\PY{p}{[}\PY{l+m+mi}{0}\PY{p}{]}
         \PY{n}{n} \PY{o}{=} \PY{n+nb}{len}\PY{p}{(}\PY{n}{softmax}\PY{p}{)}
         \PY{n+nb}{print}\PY{p}{(}\PY{l+s+s1}{\PYZsq{}}\PY{l+s+s1}{Top five softmax probabilities:}\PY{l+s+s1}{\PYZsq{}}\PY{p}{)}
         \PY{k}{for} \PY{n}{i} \PY{o+ow}{in} \PY{n+nb}{range}\PY{p}{(}\PY{n}{n}\PY{p}{)}\PY{p}{:}
             \PY{n}{txt} \PY{o}{=} \PY{l+s+s1}{\PYZsq{}}\PY{l+s+s1}{For image }\PY{l+s+si}{\PYZob{}\PYZcb{}}\PY{l+s+s1}{: }\PY{l+s+s1}{\PYZsq{}}\PY{o}{.}\PY{n}{format}\PY{p}{(}\PY{n}{i}\PY{o}{+}\PY{l+m+mi}{1}\PY{p}{)} \PY{o}{+} \PY{l+s+s1}{\PYZsq{}}\PY{l+s+s1}{, }\PY{l+s+s1}{\PYZsq{}}\PY{o}{.}\PY{n}{join}\PY{p}{(}\PY{l+s+s1}{\PYZsq{}}\PY{l+s+si}{\PYZob{}:.2f\PYZcb{}}\PY{l+s+s1}{\PYZpc{}}\PY{l+s+s1}{\PYZsq{}}\PY{o}{.}\PY{n}{format}\PY{p}{(}\PY{n}{k}\PY{p}{)} \PY{k}{for} \PY{n}{k} \PY{o+ow}{in} \PY{n}{softmax}\PY{p}{[}\PY{n}{i}\PY{p}{]}\PY{p}{)}
             \PY{c+c1}{\PYZsh{}print(\PYZsq{}For image \PYZob{}\PYZcb{} \PYZsq{}.format(i+1))}
             \PY{c+c1}{\PYZsh{}print(\PYZsq{}, \PYZsq{}.join(\PYZsq{}\PYZob{}:.2f\PYZcb{}\PYZpc{}\PYZsq{}.format(k) for k in softmax[i]))}
             \PY{n+nb}{print}\PY{p}{(}\PY{n}{txt}\PY{p}{)}
\end{Verbatim}

    \begin{Verbatim}[commandchars=\\\{\}]
Top five softmax probabilities:
For image 1: 19.66\%, 16.64\%, 14.42\%, 14.09\%, 12.59\%
For image 2: 32.42\%, 29.00\%, 25.96\%, 23.86\%, 22.89\%
For image 3: 52.91\%, 23.16\%, 11.99\%, 11.52\%, 9.34\%
For image 4: 52.65\%, 39.99\%, 24.71\%, 21.92\%, 11.52\%
For image 5: 39.69\%, 36.93\%, 23.38\%, 22.65\%, 9.82\%

    \end{Verbatim}

    \subsubsection{Project Writeup}\label{project-writeup}

Once you have completed the code implementation, document your results
in a project writeup using this
\href{https://github.com/udacity/CarND-Traffic-Sign-Classifier-Project/blob/master/writeup_template.md}{template}
as a guide. The writeup can be in a markdown or pdf file.

    \section{Traffic Sign Recognition}\label{traffic-sign-recognition}

\subsection{Writeup}\label{writeup}

Build a Traffic Sign Recognition Project

The goals / steps of this project are the following:

\begin{verbatim}
Load the data set (see below for links to the project data set)
Explore, summarize and visualize the data set
Design, train and test a model architecture
Use the model to make predictions on new images
Analyze the softmax probabilities of the new images
Summarize the results with a written report
\end{verbatim}

Rubric Points

Here I will consider the rubric points individually and describe how I
addressed each point in my implementation.

Writeup / README

\paragraph{1. Provide a Writeup / README that includes all the rubric
points and how you addressed each one. You can submit your writeup as
markdown or pdf. You can use this template as a guide for writing the
report. The submission includes the project
code.}\label{provide-a-writeup-readme-that-includes-all-the-rubric-points-and-how-you-addressed-each-one.-you-can-submit-your-writeup-as-markdown-or-pdf.-you-can-use-this-template-as-a-guide-for-writing-the-report.-the-submission-includes-the-project-code.}

Here is a link to my project code:

\subsubsection{Data Set Summary \&
Exploration}\label{data-set-summary-exploration}

\paragraph{1. Provide a basic summary of the data set. In the code, the
analysis should be done using python, numpy and/or pandas methods rather
than hardcoding results
manually.}\label{provide-a-basic-summary-of-the-data-set.-in-the-code-the-analysis-should-be-done-using-python-numpy-andor-pandas-methods-rather-than-hardcoding-results-manually.}

I used the pandas library to calculate summary statistics of the traffic
signs data set:

\begin{verbatim}
The size of training set is 34799
The size of the validation set is 4410
The size of test set is 12630
The shape of a traffic sign image is (32, 32, 3)
The number of unique classes/labels in the data set is 43
\end{verbatim}

\paragraph{2. Include an exploratory visualization of the
dataset.}\label{include-an-exploratory-visualization-of-the-dataset.}

Here is an exploratory visualization of the data set. It is a bar chart
showing how the data classification is distributed.

\begin{figure}
\centering
\includegraphics{histogram.png}
\caption{histogram}
\end{figure}

\subsubsection{Design and Test a Model
Architecture}\label{design-and-test-a-model-architecture}

\paragraph{1. Describe how you preprocessed the image data. What
techniques were chosen and why did you choose these techniques? Consider
including images showing the output of each preprocessing technique.
Pre-processing refers to techniques such as converting to grayscale,
normalization, etc. (OPTIONAL: As described in the "Stand Out
Suggestions" part of the rubric, if you generated additional data for
training, describe why you decided to generate additional data, how you
generated the data, and provide example images of the additional data.
Then describe the characteristics of the augmented training set like
number of images in the set, number of images for each class,
etc.)}\label{describe-how-you-preprocessed-the-image-data.-what-techniques-were-chosen-and-why-did-you-choose-these-techniques-consider-including-images-showing-the-output-of-each-preprocessing-technique.-pre-processing-refers-to-techniques-such-as-converting-to-grayscale-normalization-etc.-optional-as-described-in-the-stand-out-suggestions-part-of-the-rubric-if-you-generated-additional-data-for-training-describe-why-you-decided-to-generate-additional-data-how-you-generated-the-data-and-provide-example-images-of-the-additional-data.-then-describe-the-characteristics-of-the-augmented-training-set-like-number-of-images-in-the-set-number-of-images-for-each-class-etc.}

As a first step, I decided to convert the images to grayscale because
...

Here is an example of a traffic sign image before and after grayscaling.

alt text

As a last step, I normalized the image data because ...

I decided to generate additional data because ...

To add more data to the the data set, I used the following techniques
because ...

Here is an example of an original image and an augmented image:

alt text

The difference between the original data set and the augmented data set
is the following ...

\paragraph{2. Describe what your final model architecture looks like
including model type, layers, layer sizes, connectivity, etc.) Consider
including a diagram and/or table describing the final
model.}\label{describe-what-your-final-model-architecture-looks-like-including-model-type-layers-layer-sizes-connectivity-etc.-consider-including-a-diagram-andor-table-describing-the-final-model.}

My final model consisted of the following layers: Layer Description
Input 32x32x3 RGB image Convolution 3x3 1x1 stride, same padding,
outputs 32x32x64 RELU\\
Max pooling 2x2 stride, outputs 16x16x64 Convolution 3x3 etc. Fully
connected etc. Softmax etc.

\paragraph{3. Describe how you trained your model. The discussion can
include the type of optimizer, the batch size, number of epochs and any
hyperparameters such as learning
rate.}\label{describe-how-you-trained-your-model.-the-discussion-can-include-the-type-of-optimizer-the-batch-size-number-of-epochs-and-any-hyperparameters-such-as-learning-rate.}

To train the model, I used an ....

\paragraph{4. Describe the approach taken for finding a solution and
getting the validation set accuracy to be at least 0.93. Include in the
discussion the results on the training, validation and test sets and
where in the code these were calculated. Your approach may have been an
iterative process, in which case, outline the steps you took to get to
the final solution and why you chose those steps. Perhaps your solution
involved an already well known implementation or architecture. In this
case, discuss why you think the architecture is suitable for the current
problem.}\label{describe-the-approach-taken-for-finding-a-solution-and-getting-the-validation-set-accuracy-to-be-at-least-0.93.-include-in-the-discussion-the-results-on-the-training-validation-and-test-sets-and-where-in-the-code-these-were-calculated.-your-approach-may-have-been-an-iterative-process-in-which-case-outline-the-steps-you-took-to-get-to-the-final-solution-and-why-you-chose-those-steps.-perhaps-your-solution-involved-an-already-well-known-implementation-or-architecture.-in-this-case-discuss-why-you-think-the-architecture-is-suitable-for-the-current-problem.}

My final model results were:

\begin{verbatim}
training set accuracy of ?
validation set accuracy of ?
test set accuracy of ?
\end{verbatim}

If an iterative approach was chosen:

\begin{verbatim}
What was the first architecture that was tried and why was it chosen?
What were some problems with the initial architecture?
How was the architecture adjusted and why was it adjusted? Typical adjustments could include choosing a different model architecture, adding or taking away layers (pooling, dropout, convolution, etc), using an activation function or changing the activation function. One common justification for adjusting an architecture would be due to overfitting or underfitting. A high accuracy on the training set but low accuracy on the validation set indicates over fitting; a low accuracy on both sets indicates under fitting.
Which parameters were tuned? How were they adjusted and why?
What are some of the important design choices and why were they chosen? For example, why might a convolution layer work well with this problem? How might a dropout layer help with creating a successful model?
\end{verbatim}

If a well known architecture was chosen:

\begin{verbatim}
What architecture was chosen?
Why did you believe it would be relevant to the traffic sign application?
How does the final model's accuracy on the training, validation and test set provide evidence that the model is working well?
\end{verbatim}

\subsubsection{Test a Model on New
Images}\label{test-a-model-on-new-images}

\paragraph{1. Choose five German traffic signs found on the web and
provide them in the report. For each image, discuss what quality or
qualities might be difficult to
classify.}\label{choose-five-german-traffic-signs-found-on-the-web-and-provide-them-in-the-report.-for-each-image-discuss-what-quality-or-qualities-might-be-difficult-to-classify.}

Here are five German traffic signs that I found on the web:

alt text alt text alt text alt text alt text

The first image might be difficult to classify because ...

\paragraph{2. Discuss the model's predictions on these new traffic signs
and compare the results to predicting on the test set. At a minimum,
discuss what the predictions were, the accuracy on these new
predictions, and compare the accuracy to the accuracy on the test set
(OPTIONAL: Discuss the results in more detail as described in the "Stand
Out Suggestions" part of the
rubric).}\label{discuss-the-models-predictions-on-these-new-traffic-signs-and-compare-the-results-to-predicting-on-the-test-set.-at-a-minimum-discuss-what-the-predictions-were-the-accuracy-on-these-new-predictions-and-compare-the-accuracy-to-the-accuracy-on-the-test-set-optional-discuss-the-results-in-more-detail-as-described-in-the-stand-out-suggestions-part-of-the-rubric.}

Here are the results of the prediction: Image Prediction Stop Sign Stop
sign U-turn U-turn Yield Yield 100 km/h Bumpy Road Slippery Road
Slippery Road

The model was able to correctly guess 4 of the 5 traffic signs, which
gives an accuracy of 80\%. This compares favorably to the accuracy on
the test set of ...

\paragraph{3. Describe how certain the model is when predicting on each
of the five new images by looking at the softmax probabilities for each
prediction. Provide the top 5 softmax probabilities for each image along
with the sign type of each probability. (OPTIONAL: as described in the
"Stand Out Suggestions" part of the rubric, visualizations can also be
provided such as bar
charts)}\label{describe-how-certain-the-model-is-when-predicting-on-each-of-the-five-new-images-by-looking-at-the-softmax-probabilities-for-each-prediction.-provide-the-top-5-softmax-probabilities-for-each-image-along-with-the-sign-type-of-each-probability.-optional-as-described-in-the-stand-out-suggestions-part-of-the-rubric-visualizations-can-also-be-provided-such-as-bar-charts}

The code for making predictions on my final model is located in the 11th
cell of the Ipython notebook.

For the first image, the model is relatively sure that this is a stop
sign (probability of 0.6), and the image does contain a stop sign. The
top five soft max probabilities were Probability Prediction .60 Stop
sign .20 U-turn .05 Yield .04 Bumpy Road .01 Slippery Road

For the second image ...

    \begin{quote}
\textbf{Note}: Once you have completed all of the code implementations
and successfully answered each question above, you may finalize your
work by exporting the iPython Notebook as an HTML document. You can do
this by using the menu above and navigating to \n", "\textbf{File
-\textgreater{} Download as -\textgreater{} HTML (.html)}. Include the
finished document along with this notebook as your submission.
\end{quote}

    \begin{center}\rule{0.5\linewidth}{\linethickness}\end{center}

\subsection{Step 4 (Optional): Visualize the Neural Network's State with
Test
Images}\label{step-4-optional-visualize-the-neural-networks-state-with-test-images}

This Section is not required to complete but acts as an additional
excersise for understaning the output of a neural network's weights.
While neural networks can be a great learning device they are often
referred to as a black box. We can understand what the weights of a
neural network look like better by plotting their feature maps. After
successfully training your neural network you can see what it's feature
maps look like by plotting the output of the network's weight layers in
response to a test stimuli image. From these plotted feature maps, it's
possible to see what characteristics of an image the network finds
interesting. For a sign, maybe the inner network feature maps react with
high activation to the sign's boundary outline or to the contrast in the
sign's painted symbol.

Provided for you below is the function code that allows you to get the
visualization output of any tensorflow weight layer you want. The inputs
to the function should be a stimuli image, one used during training or a
new one you provided, and then the tensorflow variable name that
represents the layer's state during the training process, for instance
if you wanted to see what the
\href{https://classroom.udacity.com/nanodegrees/nd013/parts/fbf77062-5703-404e-b60c-95b78b2f3f9e/modules/6df7ae49-c61c-4bb2-a23e-6527e69209ec/lessons/601ae704-1035-4287-8b11-e2c2716217ad/concepts/d4aca031-508f-4e0b-b493-e7b706120f81}{LeNet
lab's} feature maps looked like for it's second convolutional layer you
could enter conv2 as the tf\_activation variable.

For an example of what feature map outputs look like, check out NVIDIA's
results in their paper
\href{https://devblogs.nvidia.com/parallelforall/deep-learning-self-driving-cars/}{End-to-End
Deep Learning for Self-Driving Cars} in the section Visualization of
internal CNN State. NVIDIA was able to show that their network's inner
weights had high activations to road boundary lines by comparing feature
maps from an image with a clear path to one without. Try experimenting
with a similar test to show that your trained network's weights are
looking for interesting features, whether it's looking at differences in
feature maps from images with or without a sign, or even what feature
maps look like in a trained network vs a completely untrained one on the
same sign image.

Your output should look something like this (above)

    \begin{Verbatim}[commandchars=\\\{\}]
{\color{incolor}In [{\color{incolor} }]:} \PY{c+c1}{\PYZsh{}\PYZsh{}\PYZsh{} Visualize your network\PYZsq{}s feature maps here.}
        \PY{c+c1}{\PYZsh{}\PYZsh{}\PYZsh{} Feel free to use as many code cells as needed.}
        
        \PY{c+c1}{\PYZsh{} image\PYZus{}input: the test image being fed into the network to produce the feature maps}
        \PY{c+c1}{\PYZsh{} tf\PYZus{}activation: should be a tf variable name used during your training procedure that represents the calculated state of a specific weight layer}
        \PY{c+c1}{\PYZsh{} activation\PYZus{}min/max: can be used to view the activation contrast in more detail, by default matplot sets min and max to the actual min and max values of the output}
        \PY{c+c1}{\PYZsh{} plt\PYZus{}num: used to plot out multiple different weight feature map sets on the same block, just extend the plt number for each new feature map entry}
        
        \PY{k}{def} \PY{n+nf}{outputFeatureMap}\PY{p}{(}\PY{n}{image\PYZus{}input}\PY{p}{,} \PY{n}{tf\PYZus{}activation}\PY{p}{,} \PY{n}{activation\PYZus{}min}\PY{o}{=}\PY{o}{\PYZhy{}}\PY{l+m+mi}{1}\PY{p}{,} \PY{n}{activation\PYZus{}max}\PY{o}{=}\PY{o}{\PYZhy{}}\PY{l+m+mi}{1} \PY{p}{,}\PY{n}{plt\PYZus{}num}\PY{o}{=}\PY{l+m+mi}{1}\PY{p}{)}\PY{p}{:}
            \PY{c+c1}{\PYZsh{} Here make sure to preprocess your image\PYZus{}input in a way your network expects}
            \PY{c+c1}{\PYZsh{} with size, normalization, ect if needed}
            \PY{c+c1}{\PYZsh{} image\PYZus{}input =}
            \PY{c+c1}{\PYZsh{} Note: x should be the same name as your network\PYZsq{}s tensorflow data placeholder variable}
            \PY{c+c1}{\PYZsh{} If you get an error tf\PYZus{}activation is not defined it may be having trouble accessing the variable from inside a function}
            \PY{n}{activation} \PY{o}{=} \PY{n}{tf\PYZus{}activation}\PY{o}{.}\PY{n}{eval}\PY{p}{(}\PY{n}{session}\PY{o}{=}\PY{n}{sess}\PY{p}{,}\PY{n}{feed\PYZus{}dict}\PY{o}{=}\PY{p}{\PYZob{}}\PY{n}{x} \PY{p}{:} \PY{n}{image\PYZus{}input}\PY{p}{\PYZcb{}}\PY{p}{)}
            \PY{n}{featuremaps} \PY{o}{=} \PY{n}{activation}\PY{o}{.}\PY{n}{shape}\PY{p}{[}\PY{l+m+mi}{3}\PY{p}{]}
            \PY{n}{plt}\PY{o}{.}\PY{n}{figure}\PY{p}{(}\PY{n}{plt\PYZus{}num}\PY{p}{,} \PY{n}{figsize}\PY{o}{=}\PY{p}{(}\PY{l+m+mi}{15}\PY{p}{,}\PY{l+m+mi}{15}\PY{p}{)}\PY{p}{)}
            \PY{k}{for} \PY{n}{featuremap} \PY{o+ow}{in} \PY{n+nb}{range}\PY{p}{(}\PY{n}{featuremaps}\PY{p}{)}\PY{p}{:}
                \PY{n}{plt}\PY{o}{.}\PY{n}{subplot}\PY{p}{(}\PY{l+m+mi}{6}\PY{p}{,}\PY{l+m+mi}{8}\PY{p}{,} \PY{n}{featuremap}\PY{o}{+}\PY{l+m+mi}{1}\PY{p}{)} \PY{c+c1}{\PYZsh{} sets the number of feature maps to show on each row and column}
                \PY{n}{plt}\PY{o}{.}\PY{n}{title}\PY{p}{(}\PY{l+s+s1}{\PYZsq{}}\PY{l+s+s1}{FeatureMap }\PY{l+s+s1}{\PYZsq{}} \PY{o}{+} \PY{n+nb}{str}\PY{p}{(}\PY{n}{featuremap}\PY{p}{)}\PY{p}{)} \PY{c+c1}{\PYZsh{} displays the feature map number}
                \PY{k}{if} \PY{n}{activation\PYZus{}min} \PY{o}{!=} \PY{o}{\PYZhy{}}\PY{l+m+mi}{1} \PY{o}{\PYZam{}} \PY{n}{activation\PYZus{}max} \PY{o}{!=} \PY{o}{\PYZhy{}}\PY{l+m+mi}{1}\PY{p}{:}
                    \PY{n}{plt}\PY{o}{.}\PY{n}{imshow}\PY{p}{(}\PY{n}{activation}\PY{p}{[}\PY{l+m+mi}{0}\PY{p}{,}\PY{p}{:}\PY{p}{,}\PY{p}{:}\PY{p}{,} \PY{n}{featuremap}\PY{p}{]}\PY{p}{,} \PY{n}{interpolation}\PY{o}{=}\PY{l+s+s2}{\PYZdq{}}\PY{l+s+s2}{nearest}\PY{l+s+s2}{\PYZdq{}}\PY{p}{,} \PY{n}{vmin} \PY{o}{=}\PY{n}{activation\PYZus{}min}\PY{p}{,} \PY{n}{vmax}\PY{o}{=}\PY{n}{activation\PYZus{}max}\PY{p}{,} \PY{n}{cmap}\PY{o}{=}\PY{l+s+s2}{\PYZdq{}}\PY{l+s+s2}{gray}\PY{l+s+s2}{\PYZdq{}}\PY{p}{)}
                \PY{k}{elif} \PY{n}{activation\PYZus{}max} \PY{o}{!=} \PY{o}{\PYZhy{}}\PY{l+m+mi}{1}\PY{p}{:}
                    \PY{n}{plt}\PY{o}{.}\PY{n}{imshow}\PY{p}{(}\PY{n}{activation}\PY{p}{[}\PY{l+m+mi}{0}\PY{p}{,}\PY{p}{:}\PY{p}{,}\PY{p}{:}\PY{p}{,} \PY{n}{featuremap}\PY{p}{]}\PY{p}{,} \PY{n}{interpolation}\PY{o}{=}\PY{l+s+s2}{\PYZdq{}}\PY{l+s+s2}{nearest}\PY{l+s+s2}{\PYZdq{}}\PY{p}{,} \PY{n}{vmax}\PY{o}{=}\PY{n}{activation\PYZus{}max}\PY{p}{,} \PY{n}{cmap}\PY{o}{=}\PY{l+s+s2}{\PYZdq{}}\PY{l+s+s2}{gray}\PY{l+s+s2}{\PYZdq{}}\PY{p}{)}
                \PY{k}{elif} \PY{n}{activation\PYZus{}min} \PY{o}{!=}\PY{o}{\PYZhy{}}\PY{l+m+mi}{1}\PY{p}{:}
                    \PY{n}{plt}\PY{o}{.}\PY{n}{imshow}\PY{p}{(}\PY{n}{activation}\PY{p}{[}\PY{l+m+mi}{0}\PY{p}{,}\PY{p}{:}\PY{p}{,}\PY{p}{:}\PY{p}{,} \PY{n}{featuremap}\PY{p}{]}\PY{p}{,} \PY{n}{interpolation}\PY{o}{=}\PY{l+s+s2}{\PYZdq{}}\PY{l+s+s2}{nearest}\PY{l+s+s2}{\PYZdq{}}\PY{p}{,} \PY{n}{vmin}\PY{o}{=}\PY{n}{activation\PYZus{}min}\PY{p}{,} \PY{n}{cmap}\PY{o}{=}\PY{l+s+s2}{\PYZdq{}}\PY{l+s+s2}{gray}\PY{l+s+s2}{\PYZdq{}}\PY{p}{)}
                \PY{k}{else}\PY{p}{:}
                    \PY{n}{plt}\PY{o}{.}\PY{n}{imshow}\PY{p}{(}\PY{n}{activation}\PY{p}{[}\PY{l+m+mi}{0}\PY{p}{,}\PY{p}{:}\PY{p}{,}\PY{p}{:}\PY{p}{,} \PY{n}{featuremap}\PY{p}{]}\PY{p}{,} \PY{n}{interpolation}\PY{o}{=}\PY{l+s+s2}{\PYZdq{}}\PY{l+s+s2}{nearest}\PY{l+s+s2}{\PYZdq{}}\PY{p}{,} \PY{n}{cmap}\PY{o}{=}\PY{l+s+s2}{\PYZdq{}}\PY{l+s+s2}{gray}\PY{l+s+s2}{\PYZdq{}}\PY{p}{)}
\end{Verbatim}


    % Add a bibliography block to the postdoc
    
    
    
    \end{document}
